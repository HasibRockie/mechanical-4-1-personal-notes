\documentclass{article}
\usepackage[margin=2cm]{geometry}
\usepackage{graphicx}
\usepackage[pages=some]{background}
\usepackage{titling}
\usepackage{tabularx}
\usepackage{tikz}
\usepackage{enumitem}

\geometry{a4paper}

\backgroundsetup{
    scale=1,
    angle=0,
    opacity=1,
    contents={%
        \includegraphics[width=\paperwidth,height=\paperheight]{institution_logo.jpg}
    }
}

\newcommand{\subtitle}[1]{
    \posttitle{
        \par\end{center}
        \begin{center}\large#1\end{center}
        \vskip0.5em}
}

\title{IPE-431}
\author{Md. Hasibul Islam}
\subtitle{MACHINE TOOLS}

\begin{document}
\begin{titlepage}
    \centering
    
    {\Huge\bfseries\maketitle}
    \textbf{Sristy Mam} \\
    \vspace{2cm}
    \includegraphics[width=8cm]{institution_logo.jpg}
    \vfill
    \vspace*{2cm}
\end{titlepage}

\tableofcontents
\pagebreak
\section{Lecture 01: Introduction} 
\hfill Date: 06/06/2023
\begin{itemize}
  \item \textbf{Conventional Machining}: Conventional machining refers to traditional manufacturing processes that involve the removal of material from a workpiece to achieve the desired shape or form. These processes typically utilize cutting tools such as drills, lathes, milling machines, or grinding machines to remove material through cutting, grinding, drilling, or similar operations. Conventional machining techniques are well-established and widely used in industries for shaping and finishing operations.

\item \textbf{Non-conventional Machining}: Non-conventional machining, also known as unconventional or advanced machining, encompasses a set of manufacturing processes that do not rely on traditional cutting tools. These processes use various methods to shape or modify materials, often involving thermal, chemical, electrical, or mechanical energy. Examples of non-conventional machining include electrical discharge machining (EDM), laser cutting, waterjet cutting, electrochemical machining (ECM), ultrasonic machining (USM), and abrasive jet machining (AJM). These techniques are particularly useful for materials that are difficult to machine using conventional methods or when intricate or complex shapes are required.

\item \textbf{Subtractive Manufacturing}: Subtractive manufacturing, also known as subtractive processes, refers to the manufacturing methods that involve the removal of material from a larger block or workpiece to create the desired shape or form. Conventional machining processes fall under the category of subtractive manufacturing. These processes selectively remove material until the final shape or dimensions are achieved. Subtractive manufacturing is commonly used in industries such as metalworking, woodworking, and plastics manufacturing, where excess material is removed to create the desired end product.

\item \textbf{Additive Manufacturing}: Additive manufacturing, also referred to as 3D printing, is a revolutionary manufacturing approach that involves the creation of three-dimensional objects by adding material layer by layer. In additive manufacturing, a digital model or design is converted into a physical object by adding successive layers of material, typically through processes like fused deposition modeling (FDM), stereolithography (SLA), selective laser sintering (SLS), or binder jetting. Additive manufacturing offers design freedom, customization, rapid prototyping, and the ability to produce complex geometries that are challenging or impossible to achieve with traditional manufacturing methods. It has found applications in various industries, including aerospace, automotive, medical, and product development. 
\end{itemize}


\subsubsection*{Books List}
\begin{itemize}
  \item Machine tools \hfill by Charnov
  \item Elements of Machines \hfill by Anwarul Azim 
\end{itemize}
\newpage

\section{Lecture 2: Topic}
\subsection*{Date: DD/MM/YYYY}

Content of the lecture goes here.

\end{document}
