\documentclass[12pt]{article} 
\usepackage[left=2cm, right=2cm, top=1.5cm, bottom=1.5cm]{geometry} 
\usepackage[pages=some]{background}
\usepackage{subcaption}
\usepackage{graphicx}
\usepackage{amsmath}
\usepackage{amssymb}


\begin{document}
\subsection*{Fluid Machineries Sessional (ME-422): Ex-3 (new sir)}
\hfill Date: 16-07-2023 
\subsubsection*{Name of the Experiment: \\a)Study and performance test of a submersible pump} 
\subsubsection*{b)Dismantling and assembling of a centrifugal pump} 
\vspace*{1cm}

    Centrifugal Pump : Where more flow rate required \\
    submersible pump : Where more head is required \\

    \subsubsection*{Characteristics of submersible pump:}
    \begin{itemize}
        \item Submerged into fluid. 
        \item No need for suction head 
        \item Under water, due to pressure $P (=h\rho g)$, no pull required. Rather it pushes water.
        \item No cavitation \& priming , unlike - centrifugal pump.
        \item Have Multiple impellers 
        \item A diffuser between two impellers. 
        \item Suction head is (0) zero. 
        \item Delivery head is high.     
        \item Flow rate at Most efficiency point in characteristic curve in known as operating point. 
    \end{itemize}

    \subsubsection*{Components of Centrifugal pump:}
    Gland packing: friction decrease, silling, and prevent water backflow.

\end{document}