\documentclass[14pt]{article} 
\usepackage[margin=3cm]{geometry}
\usepackage{graphicx}
\usepackage[pages=some]{background}
\usepackage{titling}
\usepackage{tabularx}
\usepackage{tikz}
\usepackage{subfigure}
\usepackage{textcase} % for text capitalization
\usepackage{newtxtext} % for Times New Roman font
\usepackage{multicol}

\geometry{a4paper}

\backgroundsetup{
    scale=1,
    angle=0,
    opacity=1,
    contents={%
        \includegraphics[width=\paperwidth,height=\paperheight]{institution_logo.jpg}
    }
}

\newcommand{\subtitle}[1]{
    \posttitle{
        \par\end{center}
        \begin{center}\large\MakeTextUppercase{#1}\end{center} % capitalize subtitle
        \vskip0.5em}
}

\title{IPE-431}
\author{Md. Hasibul Islam}
\subtitle{Machine Tools}
\date{}

\begin{document}
    \section{Ex-01 (a) : Study and performance test of a Pelton wheel}
    \subsection{Imrul Kayes Sir (20-08-2023)}

    \begin{multicols}{2}
        \subsubsection*{Pelton Wheel}
        \begin{itemize}
            \item Impulse turbine 
            \item High head, low discharge 
            \item axial flow 
            \item low specific speed 
            \item head : 150m + 
            \item specific speed ( 8 $\sim$ 27)
        \end{itemize}
    
    \subsubsection{Why specific speed?}
    Can compared with all turbines in a generalized way.
    
    \subsection{Important Points}
    \begin{itemize}
        \item Survo mechanism : automatic control 
        \item delfector 
        \item Operating curve : Q will vary, N will be constant 
        \item Characteristic curve : N will vary, Q will be constant
        \item Mechanical power = Torque $\times \omega$ 
        \item Torque: measured by hydraulic dynamometer 
        \item angular velocity: measured by tachometer 
        \item Input power = $Q \gamma H$ 
        \item Q is measured by orifice meter (manometer height difference) 
        \item H is measured by pressure gauge (with pump by bernoulli eqn, where v = 0, Z = 0) and psi to m conversion is needed.
        \item Pressure head = $\frac{P}{\gamma} + Z$ , where Z is the height difference of the gauge from the ammeter readings.  
    \end{itemize}

    \section{Ex-01 (b) : Dismantling of hermatically sealed compressor}
    \begin{itemize}
        \item Spring: Vibration absorb 
        \item Stator + rotor 
        \item stator : create magnetic field intensity 
        \item coil pickup tube : shaft like parts 
        \item Rotor $\rightarrow$ Crank $\rightarrow$ piston
        \item a chargin port
        \item 1 suction chamber, 2 delivery chamber : backup refrigerant 
        \item Suction chamber is only one, as pressure is less. so less space required 
        \item Delivery chamber is two, as pressure is high. so more space required
    \end{itemize}

    \end{multicols}
\end{document}
