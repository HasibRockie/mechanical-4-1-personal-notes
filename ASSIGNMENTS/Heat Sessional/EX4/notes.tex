\documentclass{article}
\usepackage[margin=2cm]{geometry}
\usepackage{graphicx}
\usepackage[pages=some]{background}
\usepackage{titling}
\usepackage{tabularx}
\usepackage{tikz}
\usepackage{subfigure}
\usepackage{multicol}
\usepackage{caption}
\usepackage{amsmath}
\usepackage{amssymb}

\geometry{a4paper}

\backgroundsetup{
    scale=1,
    angle=0,
    opacity=1,
    contents={%
        \includegraphics[width=\paperwidth,height=\paperheight]{institution_logo.jpg}
    }
}

\newcommand{\subtitle}[1]{
    \posttitle{
        \par\end{center}
        \begin{center}\large#1\end{center}
        \vskip0.5em}
}

\title{ME-418}
\author{Md. Hasibul Islam}
\subtitle{HEAT ENGINE SESSIONAL}

\begin{document}
\begin{titlepage}
    \centering
    
    {\Huge\bfseries\maketitle}
    \vspace{2cm}
    \includegraphics[width=8cm]{institution_logo.jpg}
    \vfill
    \vspace*{2cm}
\end{titlepage}

\tableofcontents 
\pagebreak

\section{Experiment 04: PERFORMANCE TEST OF A PETROL ENGINE AT WIDE
OPEN THROTTLE (WOT) CONDITION \\ (Ehsan Sir)} 
\hfill Date: 15/07/2023

\begin{multicols}{2}
  \begin{itemize}
    \item Accelerator: Change the position of throttle valve 
    \item Main power control mechanism $\rightarrow$ throttle 
    \item Crank-piston mechanism $\rightarrow$ rotational torque (created by the push of pressure)
    \item Mechanical power: Torque $\times$ Angular velocity
    \item If throttle opens, Torque $\uparrow$, speed $\uparrow$ 
    \item Variable speed engine. 3 ways:-\\ 
    (a) Torque same, speed different,\\
    (b) Torque different, speed same,\\
    (c) Torque different, speed different. (in this experiment, this is the case. Also automotive industries follow this.)
    \item Without high speed, we can't get high power.
    \item Rated power: The maximum power that can be operated by machine. 
    \item BSFC: Brake specific fuel consumption. (Fuel consumption per unit power, unit - kg/kW-hr or g/hp-hr)
    \item The less the BSFC, the better the engine efficiency.
    \item BSFC is measured mass basis, not volume basis. Just to maintain consistancy in all weather or place condition. 
    \item To get the mass basis, have to multiply specific gravity with volume basis.
    \item Dynamometer: A device to measure torque. In real life, it is not used. It is used to check before actual action. Only for performance evaluation. 
    \item Types of Dynamometer:\\
    a) Mechanical brake dynamometer \\
    b) Hydraulic dynamometer / Water brake dynamometer \\
    c) eddie current dynamometer \\
    \item Hydraulic dynamometer is more effective than others, taking less space. 
    \item off load / idle condition: When there is really no load, then there will be no power. When there exist something to comsume power, then it is called on load condition.  
    \item Brake $\rightarrow$ Mechanical Energy $\rightarrow$ Friction $\rightarrow$ Heat 
    \item Viscosity responsible for braking in water brake. 
    \item High speed, high braking power or high resistance.
    \item In this experiment: we won't allow to cross the temperature more than 60°. As boiling doesn't occur. Will remove the hot water.
    \item Variable load: By changing the water level. Depending on water level, resistance will also vary. 
    \item the more the horse power, more heat or temperature will be produced. 
    \item Will change the orientation of blade, by lead screw mechanism. 
    \item $Q = \dot{m} C_v dT$\\
    a) $Q$ = heat energy\\
    b) $\dot{m}$ = mass flow rate. $Q$ Can be changed by varying $\dot{m}$\\
    c) $C_v$ = specific heat capacity at constant volume. Constant\\
    d) $dT$ = change in temperature. Limited to 60°C\\
    \item Engine: 2 power.\\
    a) Indicated power: $W = \int P  \,dv $. Inside cylinder.\\
    b) Brake power: Indicated power - Internal loss = output power = Torque $\times$ Speed\\
    \item Speed measure: Magnetic induction tachometer. Magnetic flux intensity mechanism. 
    \item Intensity depends on: materials and magnetic field.
    \item Speed: There are some grooves (khaj) in shaft. so distance varies. As a result, although having the same magnetic field, magnetic intensity differs. By faraday law, voltage will be induced in milivolt scale. From there, rotation speed can be measured.
    \item Torque: Measured indirectly from reaction torque. Because, directly torque measing from a moving object is not so easy. We engine will start, due to reaction, an opposite sudden move of casing will be generated. From there, reaction torque is measured. 
    \item Here, first we measure force. The transducer which can measure force in Load cell. Here, whitstone bridge principle is used. 
    \item After calculating force, from calibration chart equivalent load in calculated. 
    \item Then, torque is calculated from the equivalent load and radial distance.
    \item $hp = \frac{W\times N}{3000}$, given my the machine manufacturer to easily calculate the horse power.
    \item Derating: When the engine is not in good condition, then the power will be less than the rated power.
    \item BS5514: 100 KPa [atm pressure], 27°C, 60\% humidity.
    \item $\alpha, \beta$ = comparison between local condition with ISO condition. Depending on environmental condition, performance can vary 2-5\%. These values calculated from the book provided by sir. 
  \end{itemize}
\end{multicols}

\hrulefill \textbf{Follow Lab sheet also.}\\
\hrulefill
\end{document}
