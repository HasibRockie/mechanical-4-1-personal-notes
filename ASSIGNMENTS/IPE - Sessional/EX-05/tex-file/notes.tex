\documentclass{article}
\usepackage[margin=2cm]{geometry}
\usepackage{graphicx}
\usepackage[pages=some]{background}
\usepackage{titling}
\usepackage{tabularx}
\usepackage{tikz}
\usepackage{subfigure}
\usepackage{multicol}
\usepackage{caption}
\usepackage{amsmath}
\usepackage{amssymb}

\geometry{a4paper}

\backgroundsetup{
    scale=1,
    angle=0,
    opacity=1,
    contents={%
        \includegraphics[width=\paperwidth,height=\paperheight]{institution_logo.jpg}
    }
}

\newcommand{\subtitle}[1]{
    \posttitle{
        \par\end{center}
        \begin{center}\large#1\end{center}
        \vskip0.5em}
}

\title{IPE-432}
\author{Md. Hasibul Islam}
\subtitle{MACHINE TOOLS SESSONAL}

\begin{document}
\begin{titlepage}
    \centering
    
    {\Huge\bfseries\maketitle}
    \vspace{2cm}
    \includegraphics[width=8cm]{institution_logo.jpg}
    \vfill
    \vspace*{2cm}
\end{titlepage}

\tableofcontents 
\hrulefill

\section{Experiment 05: Study of Gear Shaper\\ (Kaniz Maam)} 
\hfill Date: 19/08/2023

\begin{multicols}{2}
  There are two methods: \\
  \begin{enumerate}
    \item Forming
    \item Generating (in this experiment)
  \end{enumerate}
  \subsection{Advantages of Generating}
  \begin{itemize}
    \item Involute profile 
    \item speed \& motion control 
    \item ex - gear shaper, gear hober 
  \end{itemize}
  \subsection{Important Points}
  \begin{itemize}
    \item in this experiment: cutter $\rightarrow$ gear , work piece $\rightarrow$ Gearblank 
    \item Indexing: dividing equally any cylindrical or circular objects 
    \item Automatic indexing is used in this experiment. That means, for 1 revolution of cutter, there will be 1 revolution of gear blank. 
    \item Motion is maintained through change gear. 
    \item Cutter: $Z_c$ and work : $Z$ 
    \item $\frac{1}{Z_c} \times$ gear cutter = $\frac{1}{Z} \times$ gear blank 
  \end{itemize}

  \subsection{Depth of Cut}
  \begin{itemize}
    \item one pass : cut a single teeth at a time. More friction, heat generation \& gear cutter may break 
    \item multi pass : cutting teeth by step by step in multiple pass. Time consuming. 
  \end{itemize}

  \subsection{Motions : 5 motions} 
  \begin{itemize}
    \item Principle motion : Reciprocating motion 
    \item Auxiliary motion : Rotating motion of gear cutter
    \item Auxiliary motion : Rotating motion of gear blank 
    \item Auxiliary motion : Radial in feed motion 
    \item Auxiliary motion : Withdrawal motion 
  \end{itemize}
  
  \subsubsection{Reciprocating Motion}
  Fly wheel $\rightarrow$ Rack \& pinion $\rightarrow$ Shaft (Horizontal spline shaft) $\rightarrow$ Shaft (vertical spline shaft) $\rightarrow$ Reciprocating motion 
  
  \begin{itemize}
    \item Motion in cutter.
    \item Cutting stroke : material will remove.
    \item Return stroke : no material will remove.
    \item Withdrawal motions helps not to cut in return stroke. 
  \end{itemize}
  
  \subsubsection{Rotating Motion of Gear Cutter}
  Motor $\rightarrow$ Pulley $\rightarrow$ Sprocket $\rightarrow$ Worm screw $\rightarrow$ Worm wheel $\rightarrow$ Bevel gear $\rightarrow$ Change gear $\rightarrow$ worm wheel $\rightarrow$ Cutter spindle $\rightarrow$ Cutter
  
  \subsubsection{Rotating Motion of Gear Blank}
  ( Motor $\rightarrow$ Pulley $\rightarrow$ Sprocket $\rightarrow$ Worm screw $\rightarrow$ Worm wheel $\rightarrow$ Bevel gear $\rightarrow$ Change gear ) $\rightarrow$ Worm screw $\rightarrow$ Worm wheel $\rightarrow$ gear blank 
  
  \begin{itemize}
    \item Change gear : A, B, C, D 
    \item C : $Z_c$ (teeth of cutting gear) 
  \end{itemize}
  
  \subsubsection{Radial In feed Motion}
  Change gear $\rightarrow$ cam $\rightarrow$ in feed motion. 

  \begin{itemize}
    \item From table: A, B : find out feed per stroke 
  \end{itemize}

  \textbf{Follow Lab sheet also.}\\
\end{multicols}
\end{document}
