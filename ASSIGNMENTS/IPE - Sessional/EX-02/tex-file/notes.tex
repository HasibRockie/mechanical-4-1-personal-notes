\documentclass{article}
\usepackage[margin=2cm]{geometry}
\usepackage{graphicx}
\usepackage[pages=some]{background}
\usepackage{titling}
\usepackage{tabularx}
\usepackage{tikz}
\usepackage{subfigure}
\usepackage{multicol}
\usepackage{caption}
\usepackage{amsmath}
\usepackage{amssymb}

\geometry{a4paper}

\backgroundsetup{
    scale=1,
    angle=0,
    opacity=1,
    contents={%
        \includegraphics[width=\paperwidth,height=\paperheight]{institution_logo.jpg}
    }
}

\newcommand{\subtitle}[1]{
    \posttitle{
        \par\end{center}
        \begin{center}\large#1\end{center}
        \vskip0.5em}
}

\title{IPE-432}
\author{Md. Hasibul Islam}
\subtitle{MACHINE TOOLS SESSONAL}

\begin{document}
\begin{titlepage}
    \centering
    
    {\Huge\bfseries\maketitle}
    \vspace{2cm}
    \includegraphics[width=8cm]{institution_logo.jpg}
    \vfill
    \vspace*{2cm}
\end{titlepage}

\tableofcontents
\pagebreak
\section{Experiment 02: Study Milling Machine \& Dividing Head\\ (Rashik Sir)} 
\hfill Date: 17/06/2023

\begin{multicols}{2}
\subsubsection*{Why this milling machine is called "universal milling machine"?}
A milling machine who has these three characteristics are counted as universal milling machine - 
\begin{itemize}
  \item The axis motion - 
    \begin{enumerate}
      \item z axis or Longitudinal motion 
      \item x axis or Cross motion 
      \item y axis or Vertical motion 
    \end{enumerate}
  \item Sqivel Plate (make an angle for incline feed)
  \item Dividing Head 
\end{itemize}

\subsubsection*{Why it is called knee \& column type? what does it signify?}
The milling machine has a knee and a column. Knee bears the weight and give support. whereas column transmit powers. That is analogous to human body, leg and spinal cord. 

\subsubsection*{How many guideways? and their classifications?}
There are 3 guide ways. They are - 
\begin{enumerate}
  \item Dove Tail Guide way
  \item Flat or Rectangular Guide way
  \item Cylindrical guide way  
\end{enumerate}

\subsubsection*{Dove Tail Guide Way}
It looks like the tail of dove. that's why the name is given.
4 Dove tail in milling machine. 
\begin{enumerate}
  \item In between of overarm \& arbor support 
  \item In between of table \& saddle (longitudinal motion)
  \item In between of saddle \& knee (cross or x-directional motion) 
  \item In between of knee \& column 
\end{enumerate}

\subsubsection*{Flat \& Rectangular Guideway}
They are normally with the stoppers. 3 flat \& Rectangular guideway in milling machine. 
\begin{itemize}
  \item Stopper with table 
  \item Stopper with Saddle 
  \item Stopper with column 
\end{itemize}

\subsubsection*{Cylindrical Guideway}
Cylindrical guideway situated with arbor \& arbor support. 

\subsubsection*{Milling Operations}
There are 3 types of milling operations in milling machine: 
\begin{enumerate}
  \item Peripheral Milling: Known as conventional milling, cut materials from the periphery
  \item Face milling: cut materials with the face. 
  \item End milling: cut materials with cutter's teeth. 
\end{enumerate}

\subsubsection*{Cutting Strategies}
\begin{enumerate} 
  \item Upmilling : Feed motion \& cutter motion are opposite in direction 
  \item Downmilling : Feed motion \& cutter motion in same direction 
\end{enumerate}

\subsection*{Dividing Head}
\subsubsection*{Indexing}
The main function of dividing head is to equally divide a circular or cylindrical object. There are 3 types of indexing.
\begin{enumerate}
  \item Simple Indexing 
  \item Differential Indexing 
  \item Cutting Helical Gear 
\end{enumerate}

\subsubsection*{Name of different parts}
Some parts of dividing head are - 
\textbf{Index Plate, Index crank, Index Crank Handle, Index Pin, Change Gear \& Inside parts (such as - some spur gears, some bevel gears, 1 work gear \& 1 worm wheel)}\\

Required Index plate rotation
\[
= \frac{\text{Gear ratio between index plate \& gear}}{\text{Number of gears to be cut}}=\frac{40}{T}
\]

\subsubsection*{Change Gear}
When number of teeth doesn't match with the index plate, then to adjust it change gear helps. For example - we want to cut 67 teeth, but nearest available index plate number is 66. So, we will choose 66 and use change gear to adjust rest. Here, we need to use \textbf{Differential Indexing} mechanism. 

\subsubsection*{Helical Gear}
Have to rotate workpiece also and have to sync with change gear, to cut gear inclined. 

\subsubsection*{Mathematical Relations}
\begin{align*}
  p_{h.g.} &= \frac{Z_o}{Z} \times 1 \times 1 \times 1 \times \frac{a}{b} \times \frac{c}{d} \times P_{l.s.}\\
  \frac{a}{b} \times \frac{c}{d} &= \frac{P_{h.g.}}{Z_o P_{l.s.}} 
\end{align*}
here, $Z_o$ = No. of Teeth of worm wheel = 40 \\
Z = No. of start of worm wheel = 1 \\ 
$P_{h.g.}$ = Lead of helical gear \\
$P_{l.s.}$ = Lead of Lead screw \\
$\checkmark$ worm wheel will rotate a single time, if worm gear rotates 40 times. \\
$\checkmark$ For the above equation, right hand side is constant. So, we have to control a,b,c \& d to get helical shape.  \\
$\checkmark$ It's mendatory to keep error under 1\%  

\subsubsection*{Set angle relation}
set angle, $\omega$ = 90° - Helix angle ($\alpha$) = Lead angle \\
again, $$\omega = \arctan \left(\frac{\pi D}{P_{h.g.}}\right)$$

$\checkmark$ set angle means the angular rotation of swivel plate. helix angle means the inclination angle in helical gear.\\
$\checkmark$ D indicates the diameter of gear blank.\\
$\checkmark$ if we know $\omega$, then we can find out helix angle and $P_{h.g.}$ by the above equations.\\
$\checkmark$ For helical gear, we need to rotate lead screq by change gear. \\
\end{multicols}
\textbf{Follow Lab sheet also.}\\
\textbf{IMPORTANT : There will be math related with this is QUIZ.}
\hrulefill
\end{document}
