\documentclass{article}
\usepackage[margin=2cm]{geometry}
\usepackage{graphicx}
\usepackage[pages=some]{background}
\usepackage{titling}
\usepackage{tabularx}
\usepackage{tikz}
\usepackage{subfigure}
\usepackage{multicol}
\usepackage{caption}
\usepackage{amsmath}
\usepackage{amssymb}

\geometry{a4paper}

\backgroundsetup{
    scale=1,
    angle=0,
    opacity=1,
    contents={%
        \includegraphics[width=\paperwidth,height=\paperheight]{institution_logo.jpg}
    }
}

\newcommand{\subtitle}[1]{
    \posttitle{
        \par\end{center}
        \begin{center}\large#1\end{center}
        \vskip0.5em}
}

\title{IPE-432}
\author{Md. Hasibul Islam}
\subtitle{MACHINE TOOLS SESSONAL}

\begin{document}
\begin{titlepage}
    \centering
    
    {\Huge\bfseries\maketitle}
    \vspace{2cm}
    \includegraphics[width=8cm]{institution_logo.jpg}
    \vfill
    \vspace*{2cm}
\end{titlepage}

\tableofcontents 
\hrulefill

\section{Experiment 03: Study of Engine Lathe\\ (New Sir)} 
\hfill Date: 15/07/2023

\begin{multicols}{2}
\subsubsection*{Power}
\begin{itemize}
  \item Spindle speed $\rightarrow$ Primary motion speed 
  \item 3 phase induction motor $\rightarrow$ V-belt pullet $\rightarrow$ Headstock (S.G.B.) 
  \item Feed rod/ Lead screw $\rightarrow$ Feed movement 
  \item Rest all $\rightarrow$ auxiliary move.
  \item S.G.B. $\rightarrow$ spindle $\rightarrow$ Jaw $\rightarrow$ Job 
  \item F.G.B. (below S.G.B.) $\rightarrow$ Lead screw / Feed rod $\rightarrow$ Motion in carriage. 
\end{itemize}

\subsubsection*{Headstock:}
Housing of SGB, and rotation transfer to the spindle. 

\subsubsection*{Change Gear Box}
Precision threading (non-standard thread)\\
F.G.B. $\rightarrow$ speed transfer \\

\textbf{Why change gear?} 
$\rightarrow$ We can easily change the gears of CGB 

Normally, 4 gears with different combination and their transmission ratio is high. 

\subsubsection*{Stepped drive system:}
We can not get continuous speed but discrete speed is possible only. 

\subsubsection*{Important Points:}
\begin{itemize}
  \item Lead screw/feed rod $\rightarrow$ automatic feed is given 
  \item In lathe machine, almost all operations can be done except - gear cutting. 
  \item Difference between turning and threading: \\Threading $\rightarrow$ high feed rate, \\Turning $\rightarrow$ low feed rate (also known as micro threading) 
  \item 2 output shaft:\\ High feed (threading) $\rightarrow$ Lead screw \\Low feed (turning) $\rightarrow$ Feed rod 
  \item Torsional Deformation happens if high \& low feed are given one after another. Thats why different feed for different rod/shaft.
  \item Apron (cover) $\rightarrow$ can convert roational motion to feed motion. Contains carriage. 
  \item Saddle : H-shaped 
  \item Cross slide [cross movement] : half nut mechanism 
  \item Swivle Plate [angular movement] 
  \item Top slide 
  \item Tail stock : Support long workpiece and attach cutting tool. Can not move cross, only horizontal. But offsetting very small angle around 4° is possible.
  \item Guideway : Guide the movement of tailstock \& carriage. [V-shaped guideway]  
  \item topslide $\rightarrow$ manual movement in cutting tool. 
  \item half nut mechanism  $\rightarrow$ 2 position. cross slide and lead screw motion. 
\end{itemize}

\subsubsection*{Taper Turning}
\begin{enumerate}
  \item Setting over tail stock [offsetting tail]
  \item 2 feed method: simultaneous longitudial (carriage) and cross slide (cross) motion. 
  \item With swivle plate angle. 
  \item With taper turning attachment. 
\end{enumerate}

How rotary motion converted to linear motion (lead screw/ feed rod)\\
Key slot and spur gear in apron. The key sets on key slot and feed rod will move with spur gear. Rack and pinion are present (Bevel gear). 

\subsubsection*{Accesories:}
\begin{itemize}
  \item Live center  $\rightarrow$ headstock 
  \item Dead center  $\rightarrow$ tailstock 
  \item Mandle  $\rightarrow$ Holds the internal hollow workpiece 
  \item 3 jaw self centered chuck [auto center]
  \item 4 jaw independent chuck [manually centering] 
  \item Faceplate  $\rightarrow$ holds the irregular shape workpiece. 
  \item Rest  $\rightarrow$ Support the small diameter workpiece, in order to prevent buckling. \\Two types:\\a)Steady rest : fixed, doesn't move. (in bed)\\Follower rest: mounted on saddle. Follow the cutting tool. 
\end{itemize}

\end{multicols}
\textbf{Follow Lab sheet also.}\\
\hrulefill
\end{document}
