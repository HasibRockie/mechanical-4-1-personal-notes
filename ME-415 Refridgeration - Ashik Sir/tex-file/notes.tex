\documentclass{article}
\usepackage[margin=2cm]{geometry}
\usepackage{graphicx}
\usepackage[pages=some]{background}
\usepackage{titling}
\usepackage{tabularx}
\usepackage{tikz}
\usepackage{forest}
\usepackage{amsmath}
\usepackage{amssymb}

\forestset{
  my box/.style={
    draw,
    rectangle,
    rounded corners,
    fill=gray!20,
    inner sep=6pt,
    minimum width=3cm % Adjust the width as needed
  }
}


\geometry{a4paper}

\backgroundsetup{
    scale=1,
    angle=0,
    opacity=1,
    contents={%
        \includegraphics[width=\paperwidth,height=\paperheight]{institution_logo.jpg}
    }
}

\newcommand{\subtitle}[1]{
    \posttitle{
        \par\end{center}
        \begin{center}\large#1\end{center}
        \vskip0.5em}
}

\title{ME-415}
\author{Md. Hasibul Islam}
\subtitle{REFRIGERATION \& BUILD MECHANICAL SYSTEMS}

\begin{document}
\begin{titlepage}
    \centering
    
    {\Huge\bfseries\maketitle}
    \textbf{Md. Ashiqur Rahman Sir} \\
    \vspace{2cm}
    \includegraphics[width=8cm]{institution_logo.jpg}
    \vfill
    \vspace*{2cm}
\end{titlepage}

\tableofcontents
\pagebreak
\section{Lecture 01: Introduction} 
\hfill Date: 06/06/2023

Build Mechanical System:
\begin{itemize}
  \item HVAC
  \item Fire Protection
  \item Plumbing
  \item Transportation System (Elevators, Escalators etc.)
\end{itemize}

\subsubsection*{HVAC}
HVAC stands for Heating, Ventilation, and Air Conditioning. It is a technology or system used to control the indoor environment, including temperature, humidity, and air quality, in residential, commercial, and industrial buildings.

Here's a brief explanation of each component of HVAC:
\begin{itemize}
  \item \textbf{Heating}: Heating systems are responsible for raising the temperature of an indoor space during colder periods. Common heating systems include furnaces, boilers, heat pumps, and electric heaters. They generate heat and distribute it throughout the building, ensuring a comfortable temperature.
  \item \textbf{Ventilation}: Ventilation is the process of exchanging or replacing indoor air with outdoor air to maintain air quality. It involves the removal of stale air, odors, and contaminants and the introduction of fresh air. Ventilation systems can be natural (through windows or vents) or mechanical (using fans or ducts). Proper ventilation helps remove pollutants, control moisture, and provide a healthy and comfortable indoor environment.
  \item \textbf{Air Conditioning}: Air conditioning refers to the cooling and dehumidification of indoor air to maintain a comfortable temperature during hot or humid weather. Air conditioning systems, commonly known as AC or HVAC units, use refrigeration principles to cool the air. They remove heat from the indoor space and release it outside, providing cool air for occupants.

\end{itemize}

HVAC systems are designed to provide thermal comfort and maintain a healthy indoor environment. They are found in residential homes, offices, schools, hospitals, shopping malls, and various other buildings. HVAC systems can be centralized, where a single system serves the entire building, or decentralized, with separate units for individual spaces or rooms.

Additionally, HVAC systems can include other components such as air filtration systems, humidity control devices, thermostats, and energy management systems to enhance comfort, efficiency, and control of the indoor environment.

\subsubsection*{Plumbing}
Plumbing refers to the system of pipes, fixtures, and devices used to supply water, gas, or other fluids and remove wastewater from buildings. It includes the installation, repair, and maintenance of the plumbing infrastructure in residential, commercial, and industrial properties. Plumbing systems ensure a reliable water supply, proper drainage, and disposal of wastewater, as well as the distribution of gas for heating and cooking. Plumbers handle the design and installation of plumbing systems, ensuring compliance with safety and building codes.

\newpage

\section{Lecture 2: Fire Protection}
\hfill Date: 13/06/2023

\subsection*{Fire dynamics}
Fire dynamics is the study of how fires start, spread, and behave. It explores ignition, combustion, fire growth, and suppression methods. Understanding fire dynamics helps with fire safety, engineering, and firefighting operations.

\subsubsection*{Flash Point}
Flash point refers to the minimum temperature at which a substance releases enough vapor to form an ignitable mixture with the air near its surface. It is a crucial parameter for determining the fire hazard potential of a material.

\subsubsection*{Fire Point}
Fire point is the temperature at which a substance produces sufficient vapors to sustain combustion once the ignition source is removed. It is the temperature at which a substance continues to burn after being ignited.

\subsubsection*{Flammability Limit}
Flammability limit, also known as the explosive range, refers to the range of concentrations of a flammable substance in a mixture with air that can support combustion. It consists of two limits: the lower flammability limit (LFL) and the upper flammability limit (UFL). Below the LFL, the mixture is too lean to ignite, while above the UFL, it is too rich to ignite. Within this range, if an ignition source is present, a fire or explosion can occur.

\subsubsection*{Some important points}
\begin{enumerate}
  \item Fire $\rightarrow$ self-sustained oxidation of fuel 
  \item 3 components are mendatory to create fire. They are - Fuel, Heat \& Oxygen 
  \item 3 components are given out as a resulf of fire. They are -  Heat, Light \& Smoke 
  \item combustion occurs when the fuel is in the \textbf{gaseous} state 
  \item fire occurance order $\rightarrow$ gaseous $\geqslant$  liquid $\geqslant$  solid
  \item The higher the surface area to mass ratio, the more the fire 
  \item Heat is the most crucial element to propagate or sustain the fire. Because heat can produce further heat by burning
  \item Minimum 16\% oxygen ratio is required to sustain flamming fire.
  \item The higher the concentration of $O_2$, the higher the fire rate 
  \item smoke is most crucial in fire accidents. About 75\% people die because of smoke in fire accident. 
\end{enumerate}

\subsubsection*{Explosion}
\textbf{Detonation}: Detonation refers to a rapid and violent combustion process that occurs at supersonic speeds. It involves the nearly instantaneous release of energy in the form of a shock wave. The reaction front moves faster than the speed of sound, creating a highly destructive and explosive event. Detonations typically occur in highly reactive and confined environments, such as high-explosive materials.
\\

\textbf{Deflagration}: Deflagration is a relatively slower combustion process compared to detonation. It involves a subsonic flame front that propagates through a fuel-air mixture. The combustion wave spreads at a speed lower than the speed of sound, resulting in a less intense and less destructive event compared to detonation. Common examples of deflagrations include fires, most chemical explosions, and the combustion of fuels in engines.\\

\textbf{Flaming Fire}: A flaming fire is characterized by the presence of visible flames. It involves the rapid oxidation of a fuel in the presence of sufficient heat and oxygen. Flames are typically visible, and the fire releases heat, light, and often produces smoke. Flaming fires are commonly associated with open fires, such as those produced by burning wood, paper, or flammable liquids. They tend to spread quickly and are more easily extinguished by removing the fuel source or using appropriate fire suppression methods.\\

\textbf{Smoldering Fire}: A smoldering fire, on the other hand, is a slow, low-temperature combustion process without the presence of visible flames. It occurs when a material undergoes incomplete combustion due to limited oxygen availability. Smoldering fires are characterized by glowing embers or hot spots that produce smoke and heat but lack the intense flames associated with flaming fires. Smoldering fires can be challenging to detect and extinguish as they can persist for extended periods, hidden within materials such as upholstery, peat, or smoldering cigarette butts. They pose a significant risk of rekindling into a flaming fire if provided with additional oxygen and fuel.

\subsubsection*{Tenability \& it's limit}
Tenability refers to the conditions within a space that are considered safe and tolerable for occupants during a fire or other hazardous event. It relates to the ability of individuals to survive, evacuate, and be protected from the harmful effects of fire, smoke, heat, and toxic gases.

Tenability limits are the thresholds beyond which the conditions in a space become untenable and pose a significant risk to human life. These limits define the point at which the environment becomes hazardous and occupants may experience adverse health effects or be unable to survive. Common tenability limits include:

\begin{itemize}
  \item Temperature Limit: The temperature at which occupants may be at risk of burns or heat-related injuries. Specific temperature limits may vary depending on factors such as duration of exposure and the vulnerability of occupants.
  \item Visibility Limit: The point at which reduced visibility due to smoke or other factors hinders occupants' ability to navigate and evacuate safely.
  \item Toxic Gas Limit: The concentration of toxic gases, such as carbon monoxide (CO) or hydrogen cyanide (HCN), beyond which occupants may be at risk of acute or chronic health effects.
  \item Oxygen Limit: The lower limit of oxygen concentration below which occupants may experience difficulty breathing or unconsciousness.
  \item Radiant Heat Limit: The level of radiant heat flux at which occupants may sustain burns or other thermal injuries.
\end{itemize}

\subsubsection*{Classes of Fire}
\begin{enumerate}
  \item \textbf{Class A Fire}: Class A fires involve ordinary combustible materials such as wood, paper, fabric, plastics, and other common materials. These fires typically leave behind ash after burning.

  \item \textbf{Class B Fire}: Class B fires involve flammable liquids or gases such as gasoline, oil, propane, butane, solvents, and certain paints. These fires can spread rapidly and produce significant heat and flames.
  
  \item \textbf{Class C Fire}: Class C fires involve energized electrical equipment or wiring, such as appliances, electrical panels, or power tools. The key concern in a Class C fire is the potential for electrical shock, so it requires specialized approaches to extinguishing, considering the electrical hazard.
  
  \item \textbf{Class D Fire}: Class D fires involve combustible metals, including magnesium, titanium, sodium, potassium, and certain metal powders or flakes. These fires can be extremely hazardous and require specialized extinguishing agents specifically designed for suppressing metal fires.
  
  \item \textbf{Class K Fire}: Class K fires involve cooking oils and fats, commonly found in commercial kitchens and cooking facilities. These fires pose unique challenges due to the high temperatures and the potential for re-ignition. Class K fire extinguishers or specialized suppression systems are necessary to effectively extinguish them.
\end{enumerate}

\subsubsection*{Timeline analysis}
RSET (Required Safe Egress Time) is the time needed for occupants to evacuate a building safely during a fire. \\

ASET (Available Safe Egress Time) is the time available for evacuation before conditions become hazardous. \\

For safety, it is crucial that the ASET (Available Safe Egress Time) is greater than or equal to the RSET (Required Safe Egress Time). If the ASET is larger than the RSET, it provides occupants with sufficient time to evacuate before the environment becomes hazardous. This ensures a margin of safety during fire emergencies. However, if the RSET exceeds the ASET, it indicates inadequate evacuation provisions and may require adjustments to the building design, evacuation strategies, or fire protection measures to enhance occupant safety. 

$$RSET \leq  ASET$$
$$t_p + t_a + t_{rs} \leq t_u$$

where, \\$t_p$ = the time elapsed from ignition to the moment at which the fire is detected (the time of perception)\\
$t_a$ =  the time from detection/perception to the beginning of the ‘escape activity’\\
$t_{rs}$ = the time to move to a place of ‘relative safety’\\
$t_u$ = the time from ignition of the fire to the production of untenable conditions at the location under consideration (closely related to ASET)

\section{Lecture 03: CONTROL/EXTINGUISHMENT OF FIRE} 
\hfill Date: 18/06/2023

\subsubsection*{Fire Triangle}
The fire triangle represents the three essential components needed for a fire to exist. These components are fuel, heat, and oxygen. The fire triangle is often depicted as an equilateral triangle, with each side representing one of the three components. Here's a breakdown of the components:
\begin{itemize}
  \item Fuel: This refers to any material that can burn. It can be solid (wood, paper, fabric), liquid (gasoline, oil), or gas (natural gas, propane). Fuel provides the source of combustible material required for a fire.
  \item Heat: Heat is the energy that raises the temperature of the fuel to its ignition point. It can be generated through various means, such as an open flame, electrical sparks, friction, or chemical reactions.
  \item Oxygen: Oxygen is necessary for the combustion process. It is present in the air we breathe and is one of the components of the atmosphere. It supports the chemical reaction that sustains the fire.
\end{itemize}


\subsubsection*{Fire Tetrahedron} 
The fire tetrahedron is an extension of the fire triangle, incorporating a fourth component that is crucial to fire behavior. In addition to fuel, heat, and oxygen, the fire tetrahedron includes the element of a chemical chain reaction. Here's a breakdown of the components:
\begin{itemize}
  \item Fuel: Same as in the fire triangle, fuel refers to the material that can burn.
  \item Heat: Heat is the energy required to raise the fuel's temperature to the ignition point.
  \item Oxygen: Oxygen sustains the combustion process.
  \item Chemical Chain Reaction: This is the fourth component of the fire tetrahedron. It represents a self-sustaining chemical reaction that occurs once the fuel, heat, and oxygen are present in sufficient amounts. The chain reaction releases heat, which further raises the temperature of the fuel, continuing the fire.
\end{itemize}

\subsubsection*{Important Points}
\begin{itemize}
  \item If we remove one element out of four element from a fire accident, fire will be off. 
  \item fire protection → Fire Tetrehedron 
  \item fire prevention → Fire Triangle 
  \item \textbf{Smoldering}:
  Smoldering refers to a type of combustion that occurs without an open flame. It is a slow, low-temperature, and incomplete burning process that produces smoke and glowing embers. In smoldering fires, the fuel undergoes a chemical reaction and slowly oxidizes without sufficient heat to sustain a visible flame. Smoldering fires can be dangerous because they can release toxic gases and often have a tendency to reignite if not fully extinguished.
  Examples of smoldering fires include cigarettes or embers left unattended, electrical malfunctions causing overheating without an open flame, or the slow decomposition of organic materials such as peat or compost.
  
  \item \textbf{Smothering}:
  Smothering is a fire suppression technique that involves cutting off the fire's oxygen supply to extinguish or control the flames. By depriving the fire of oxygen, it disrupts the combustion process and prevents the fire from sustaining itself. Smothering can be achieved by covering the fire with a non-combustible material, such as a fire blanket, sand, dirt, or a heavy object.
  When smothering a fire, it's important to completely cover the burning material and ensure a tight seal to prevent the entry of oxygen. This technique is effective for small fires or those confined to a specific area, as it limits the fire's access to the necessary oxygen for combustion.
  
  Smothering is often used in situations where water may not be suitable or available, such as when extinguishing certain types of flammable liquid fires or fires involving electrical equipment.
\end{itemize}



\subsubsection*{How to control the FUEL}
Controlling the fuel is a crucial aspect of extinguishing a fire. Here are several methods commonly used to control and remove fuel sources to suppress or extinguish a fire:
\begin{enumerate}
  \item Removal or Separation: If the fuel source is portable or movable, the first step is to remove it from the fire. For example, in the case of a small fire involving a burning object like paper or fabric, you can physically separate the fuel from the flames using non-flammable tools or by smothering it with a fire blanket.
  \item Shutting off Fuel Sources: In situations where the fuel is a liquid or gas, shutting off the fuel source can effectively control the fire. This can be done by closing valves, turning off switches, or disconnecting fuel lines.
  \item Smothering: Smothering is a technique used to deprive the fire of oxygen and control the fuel. It involves covering the fire with a non-combustible material like a fire blanket, sand, or a heavy object, effectively smothering the flames and preventing the fire from accessing additional fuel.
  \item Cooling: Cooling the fuel can reduce its temperature below the ignition point, effectively controlling the fire. Water is commonly used for this purpose. Using water hoses, sprinkler systems, or fire extinguishers, aim the water at the base of the fire to cool the fuel and surrounding areas.
  \item Dilution: Diluting the fuel can be effective in controlling certain types of fires, especially those involving flammable liquids. For example, foam fire extinguishers create a layer of foam that separates the fuel from the air, preventing the release of flammable vapors and extinguishing the fire.
\end{enumerate}

\subsubsection*{How to control the OXYGEN}
\begin{enumerate}
  \item Make oxygen level below 16\% to stop fire propagation. For stopping smoldering, keep oxygen level below 8\%. 
  \item Smothering: Smothering involves covering the fire with a non-combustible material to cut off its oxygen supply. This prevents the fire from receiving the necessary oxygen to sustain combustion. Suitable materials for smothering include fire blankets, sand, dirt, or a heavy object like a metal lid. Place the smothering material over the flames, ensuring that it covers the entire burning area.  
  \item Sealing: By closing doors, windows, or other openings in the vicinity of the fire, you can limit the oxygen supply and reduce the fire's intensity. This method is effective for small fires in enclosed spaces, as it restricts the air circulation and reduces the oxygen available for the fire.
  
  \item Ventilation Control: In some cases, controlling the ventilation can help in extinguishing a fire. By closing vents or using dampers to restrict the air supply to the fire area, you can reduce the amount of oxygen available and limit the fire's growth. However, this method should only be used if it can be done safely, as it can cause the fire to spread or create hazardous conditions.
  
  \item Inert Gas Systems: In certain situations, specialized fire suppression systems that use inert gases, such as carbon dioxide (CO2) or nitrogen (N2), can be employed. These gases displace oxygen in the fire area, effectively reducing its concentration below the level required for combustion. These systems are often used in enclosed spaces, data centers, or areas with sensitive equipment.
\end{enumerate}
\subsubsection*{How to control the HEAT}
\begin{enumerate}
  \item Cooling:
  Cooling involves reducing the temperature of the fuel and surrounding areas to a point below the ignition temperature, thereby controlling the fire. Water is the most commonly used agent for cooling fires. Techniques for cooling include:
  \begin{itemize}
    \item   Hose Streams: Directing a steady stream of water onto the burning material or the base of the flames to absorb heat and cool the fuel.
    \item Sprinkler Systems: Automatic sprinkler systems are designed to release water when activated by heat, providing a continuous flow of water to cool the fire area.
    \item Fire Extinguishers: Some fire extinguishers, such as water-based extinguishers, utilize water to cool and suppress fires. Aim the extinguisher at the base of the fire and sweep from side to side.
  \end{itemize}
    \item Heat Reduction:
    Reducing heat sources or preventing heat buildup can help control fires. Methods to reduce heat include:
    \begin{itemize}
      \item Removal of Heat Sources: Shutting off heat-producing equipment, cutting off electrical power, or removing heat-generating materials can limit the fire's intensity and help control the flames.
      \item Ventilation: Properly ventilating the area by opening windows or using exhaust fans can remove heat and reduce the fire's intensity. This method is most effective in the early stages of a fire and should be done cautiously to prevent fire spread.
    \end{itemize}    
    \item Heat Shielding:
    Using heat-resistant barriers or shielding materials can help protect adjacent areas or prevent the spread of fire. This can involve using fire-resistant blankets, curtains, or barriers to block radiant heat and protect vulnerable surfaces or materials from igniting.  
\end{enumerate}

\subsubsection*{Inhibiting The chemical chain reaction}
$$CH_4 + O_2 \rightarrow CO_2 + H_2O + Heat$$

This reaction takes 37 steps and create many radicals. Have to remove those radicals to stop the reaction propagation. 

\subsection*{Extinguishment of Fire }
\subsubsection*{Water}
Water can be applied as fog, to increase surface area. As a result, rapid heat obsorbtion will occur. Advantages of smaller water droplets or steam:
\begin{itemize}
  \item Increased Surface Area: Smaller water droplets provide a larger surface area, which promotes faster heat transfer and cooling. This can be advantageous for extinguishing fires involving solid materials or fires with a large heat release rate.
  \item Enhanced Penetration: Smaller droplets or steam can penetrate deeper into the burning material, reaching hidden or hard-to-reach areas. This can help extinguish hidden or smoldering fires that may not be easily accessible.
  \item Vaporization: Steam can absorb a significant amount of heat during the vaporization process. As the water droplets evaporate, they absorb heat from the fire, cooling the surrounding area and suppressing the flames. 
\end{itemize}

\subsubsection*{Are water steam with smaller droplet always preferable or advantageous for fire extiguishment?}
If water droplet is very small, then it will start to convert to vapor before preventing fire. So, below a specific limit, water droplet size won't be effective. Have to maintain a certian limit. 

\subsubsection*{Limitations of water in fire:}
\begin{enumerate}
  \item Water is denser than most hydrocarbon fuels, and immiscible as well.water will NOT provide an effective cover
  for burning hydrocarbons, or mix with them. Instead, the hydrocarbons will float on top of the water, continuing to burn and possibly spread.
  \item Ineffective for certain fuel types: Water may not be suitable for extinguishing fires involving certain types of flammable liquids, such as oil, gasoline, or alcohol. Water can potentially spread these flammable liquids, leading to the fire's intensification or the risk of creating additional hazards.

  \item Electrical conductivity: Water is a conductor of electricity, so using water to extinguish electrical fires can pose a risk of electrical shock. It is crucial to de-energize the electrical source before applying water or use specialized fire extinguishers designed for electrical fires.
  
  \item Limited reach and accessibility: Water, especially from handheld fire extinguishers or hoses, has a limited reach. It may not be effective in extinguishing fires in high or hard-to-reach areas, such as ceilings, roofs, or elevated structures.
  
  \item Not suitable for certain environments: Water may not be appropriate in environments where its use can cause additional damage or hazards. For example, water can damage sensitive equipment, such as electronics or certain chemicals, or it may be unsuitable for use in environments with a risk of freezing or where water availability is limited.
  
  \item Potential for water damage: Using large quantities of water to extinguish fires can cause significant water damage to structures, belongings, or sensitive materials. This consideration is particularly relevant in environments where water damage can be costly or detrimental.
  
  \item Limited effectiveness on deep-seated or smoldering fires: Water may have difficulty reaching deep-seated fires or smoldering materials hidden within a structure. It may require significant quantities of water and longer application times to effectively penetrate and extinguish these types of fires.
\end{enumerate}

\subsubsection*{Heat Release Rate}
The heat release rate (HRR) in a fire refers to the amount of heat energy released per unit of time by the combustion process. It indicates the fire's heat output and is important for understanding fire behavior and designing fire protection systems.

\subsubsection*{$CO_2$ as fire extingusher}
Carbon dioxide (CO2) is used as a fire extinguishing agent due to its ability to displace oxygen and smother the fire. It is non-conductive, leaves no residue, and rapidly suppresses flames. CO2 is suitable for electrical fires, confined spaces, and areas with sensitive equipment. 

34\% or more $CO_2$ is required to create total flooding system and stop fire. 

\subsubsection*{Total Flooding System}
A total flooding system is a fire suppression system that fills an enclosed space with a fire suppressant agent, such as gas or CO2, to quickly extinguish fires. It activates automatically, uses adequate agent quantity, and ensures even distribution for effective suppression. Commonly used in critical facilities to minimize damage and protect occupants.

\subsubsection*{Clean Agent as fire extinguisher}
Clean agents, such as FM-200, Novec 1230, and others, are used as fire extinguishing agents in various applications. Here are some key uses and advantages of clean agents in fire extinguishment:
\begin{enumerate}
  \item Rapid and Effective Suppression: Clean agents rapidly extinguish fires by interrupting the combustion process. They can quickly reduce the heat, cool the fuel, and displace oxygen, leading to rapid fire suppression.
  \item Safe for Occupied Spaces: Clean agents are safe for use in occupied areas as they do not leave any residue or harmful byproducts. They are non-toxic and non-conductive, posing minimal risk to human health and sensitive equipment.
  \item Environmentally Friendly: Clean agents have a low environmental impact. They have zero ozone depletion potential (ODP) and low global warming potential (GWP), making them environmentally responsible choices for fire suppression.
  \item Protects Sensitive Equipment: Clean agents are commonly used in facilities with valuable or sensitive equipment, such as data centers, server rooms, control rooms, and museums. They provide effective fire protection without causing damage or corrosion to the equipment.
  \item Effective for Enclosed Spaces: Clean agents are particularly well-suited for protecting enclosed spaces, where traditional water-based systems may cause damage or be ineffective. They can rapidly fill the protected space, suppressing the fire without the need for extensive cleanup or water damage mitigation.
  \item Flexible Design and Installation: Clean agent systems can be designed and customized to fit specific applications, including the selection of the appropriate agent, discharge duration, and distribution method. They can be installed in a variety of settings, from small enclosed spaces to large industrial areas.
\end{enumerate}

\subsubsection*{Mechanism of clean agent}
\begin{itemize}
  \item Oxygen Displacement: Clean agents displace oxygen from the fire area, reducing its concentration below the level required for combustion. This inhibits the fire's ability to sustain itself. Examples of clean agents that work through oxygen displacement include carbon dioxide (CO2) and Inergen (IG-541).

  \item Interruption of Combustion Reactions: Clean agents interfere with the chemical chain reactions involved in combustion, breaking the chain and preventing the fire from spreading. This effectively inhibits the fire's progression. Examples of clean agents that interrupt combustion reactions include halocarbon agents like FM-200 (HFC-227ea) and Novec 1230 (FK-5-1-12).
  
  \item Heat Absorption: Clean agents have the ability to absorb heat energy from the fire, rapidly cooling the temperature of the surrounding fuel and the fire itself. By reducing the temperature, clean agents impede the fire's ability to sustain combustion. Clean agents that are effective in heat absorption include FM-200, Novec 1230, and other halocarbon agents.
\end{itemize}


\subsubsection*{Types of clean agent}
\begin{itemize}
  \item Halocarbon Agents:
  \begin{itemize}
    \item FM-200 (HFC-227ea)
    \item Novec 1230 (FK-5-1-12)
    \item FE-13 (HFC-23)
    \item FE-25 (HFC-125)
    \item FE-36 (HFC-236fa)
  \end{itemize}
  
  \item  Inert Gases:
  \begin{itemize}
    \item Inergen (IG-541) - Nitrogen, Argon, and Carbon Dioxide blend
    \item Argonite (IG-55) - Argon and Nitrogen blend
  \end{itemize}
  
\end{itemize}

\subsubsection*{ODP \& GWP}
\textbf{ODP - Ozone Depletion Potential}: ODP refers to the ability of a substance to deplete the ozone layer in the Earth's stratosphere. The ozone layer plays a crucial role in protecting the planet from harmful ultraviolet (UV) radiation. Substances with a high ODP have the potential to break down ozone molecules, leading to thinning of the ozone layer. This can result in increased UV radiation reaching the Earth's surface, which has adverse effects on human health and the environment.\\

\textbf{GWP - Global Warming Potential}: GWP measures the ability of a substance to trap heat in the atmosphere over a specific period compared to carbon dioxide (CO2). It is used as a metric to evaluate the contribution of different substances to climate change. The GWP of CO2 is defined as 1, and GWP values for other substances are relative to this baseline. A higher GWP indicates a stronger greenhouse effect and a greater potential for contributing to global warming.\\

When considering fire extinguishing agents, it is important to choose substances with low ODP and GWP to minimize their environmental impact. Clean agents such as FM-200, Novec 1230, and inert gases (e.g., Inergen, Argonite) are examples of clean agents with low or zero ODP and low GWP. These agents provide effective fire suppression while minimizing their impact on the ozone layer and contributing to climate change.

\subsubsection*{Other Extiguisher}
\begin{itemize}
  \item Dry Chemical Agents:
  Dry chemical agents are fire extinguishing agents that come in a dry powder form. They work by interrupting the chemical chain reaction of a fire. Common dry chemical agents include monoammonium phosphate (MAP), sodium bicarbonate, and potassium bicarbonate. They are effective on Class A, B, and C fires and can provide rapid suppression. However, they may leave a residue and can be corrosive to certain materials and electronics.
  
  \item Dry Powder:
  Dry powder fire extinguishers use fine powder, such as sodium bicarbonate or potassium bicarbonate, to smother the fire and interfere with the chemical reactions necessary for combustion. They are versatile and can be used on Class A, B, C, and electrical fires. Dry powder extinguishers are commonly used in industrial settings, vehicles, and marine applications. They are non-conductive and do not leave a residue, but they may have limited cooling properties.
  
  \item Wet Chemical Agents:
  Wet chemical agents, such as potassium acetate or potassium citrate solutions, are specifically designed for Class K fires involving cooking oils and fats. These extinguishers spray a fine mist that creates a soapy layer, forming a blanket on the burning oil or fat to prevent re-ignition. Wet chemical agents cool the fire and suppress the release of flammable vapors. They are commonly used in commercial kitchens and restaurants.
  
  \item Foam:
  Foam fire extinguishers discharge a foam mixture that acts as a blanketing and cooling agent. The foam smothers the fire, separates the fuel from oxygen, and suppresses flammable vapors. Foam extinguishers are effective on Class A and B fires. They are commonly used in industrial settings, warehouses, and areas with flammable liquids. Foam extinguishers can leave a residue and may require post-fire cleanup.

\end{itemize}

\subsubsection*{Foam}
Foam is a mix of surfectant. It reduces the surface tension of water, as a result cohesive force is decreased. Additionally add a layer of foam, that inbihit firing. 

\subsubsection*{Types of foam}
\begin{enumerate}
  \item \textbf{Aqueous Film-Forming Foam (AFFF):}
  AFFF is a foam concentrate that forms a thin aqueous film on the fuel surface, creating a barrier that prevents the release of flammable vapors and suppresses the fire. It is effective for extinguishing Class B (flammable liquid) fires. AFFF is compatible with hydrocarbon-based fuels like gasoline, diesel, and jet fuels. However, it is not designed to be used directly on fires involving polar solvents, such as alcohols and ketones, as they can break down the foam film and reignite the fire.
  
  \item \textbf{Alcohol-Resistant Foam (AR-AFFF):}
  AR-AFFF is specifically formulated to provide fire suppression capabilities for fires involving polar solvents like alcohols and ketones. These foams have a specialized additive that forms a protective barrier between the foam and the alcohol-based fuel, preventing the breakdown of the foam and allowing it to effectively extinguish the fire. AR-AFFF is compatible with both hydrocarbon-based fuels and polar solvents. It is commonly used in environments where alcohol-based fuels or polar solvents are present, such as in certain industrial processes or storage facilities.
\end{enumerate}

\subsubsection*{Limitations of Foam}
\begin{itemize}
  \item where water is not effective, there foam can not be applied. As foam is used with water typically with 1:99 or 3:97 ratio. 
  \item Ineffectiveness on Class C (energized electrical equipment), Class D (combustible metals), and Class K (cooking oils and fats) fires.
  \item Limited range and discharge time compared to other extinguishing methods.
  \item Foam residue requires post-fire cleanup.
  \item Potential for corrosion and damage to certain materials.
  \item Environmental concerns related to some foam concentrates.
  \item When vapor pressure is too high, foam won't work theke. 
\end{itemize}
\pagebreak

\section{Lecture 04: FIRE PROTECTION SYSTEMS} 
\hfill Date: 20/06/2023

\subsection*{Fire Protection System}
\subsubsection*{Classification}
1) Fire Alarm and Detection Systems, 
2) Fire Suppression Systems\\

Can also be divided into - \\
i) Active : Alarms, Sprinklers, Hose systems,
extinguishers\\
ii) Passive : The building itself – a concrete wall!\\

\subsubsection*{Occupacy Hazard Classification}
According to BNBC 2020, the hazard classification is
\begin{itemize}
  \item Light Hazard - I
  \item Light Hazard – II
  \item Ordinary Hazard - I
  \item Ordinary Hazard - II
  \item Ordinary Hazard - III
  \item Extra Hazard
\end{itemize}

\subsubsection*{Light Hazard}
\begin{itemize}
  \item Quantity and/or combustibility of contents is low
  \item Relatively low rates of heat release
  \item EXAMPLES: Hospitals, Museums, Offices, etc.
\end{itemize}

\subsubsection*{Ordinary Hazard - I}
\begin{itemize}
  \item Combustibility is low, quantity of combustibles is moderate
  \item Moderate heat release rate
  \item EXAMPLES: Laundry, beverage manufacturing Restaurant
  Service area, Bakery etc
\end{itemize}

\subsubsection*{Ordinary Hazard (Group 2)}
Quantity and combustibility of contents are moderate to high.Moderate to high heat release rate. EXAMPLES: RMG, Wood Plant, Bakery, etc.

\subsubsection*{Extra Hazard (Group 1)}
Quantity and combustibility of contents are very high. Rapidly developing fire with very high heat release rate. Little or no combustible/flammable liquid present. EXAMPLES: Sawmill, Aircraft hangers, etc.

\subsubsection*{Extra Hazard (Group 2)}
Moderate to substantial amounts of flammable or combustible
liquids. Rapidly developing fire with very high heat release rate. EXAMPLES: Plastic processing, Flammable Liquids
Spraying, Solvent Cleaning etc.

\subsection*{Fire alarm \& Detection System}
\begin{enumerate}
  \item Heat Detectors: Heat detectors are devices that sense a significant rise in temperature and trigger an alarm when a specific temperature threshold is exceeded. They are commonly used in areas where smoke or dust may cause false alarms, such as kitchens or dusty environments.

  \item Smoke Detectors: Smoke detectors are devices that detect the presence of smoke particles in the air. They can operate based on various principles, including ionization and photoelectric methods. When smoke is detected, the alarm is activated, alerting occupants to the potential fire.
  
  \item CO Detectors: CO (Carbon Monoxide) detectors are specifically designed to detect the presence of carbon monoxide gas, which is an odorless and colorless toxic gas produced by incomplete combustion. CO detectors provide an early warning of this gas, allowing people to evacuate and seek fresh air.
  
  \item Multi-sensor Detectors: Multi-sensor detectors combine multiple detection technologies, such as smoke and heat detection, into a single device. They provide enhanced fire detection capabilities by analyzing multiple parameters simultaneously, improving detection accuracy and reducing false alarms.
  
  \item Manual Call Points: Manual call points, also known as manual pull stations or fire alarm buttons, are devices installed throughout a building to allow occupants to manually initiate a fire alarm. When activated by pulling a lever or pressing a button, the alarm system is triggered, notifying others of the emergency.
\end{enumerate}

\subsubsection*{Advantages, Disadvantages \& Applications}
\begin{enumerate}
  \item Heat Detectors:
  \begin{enumerate}
    \item   Advantages:
    \begin{enumerate}
      \item Effective for detecting fires without the presence of smoke, such as in areas with dust or where smoke detection may be prone to false alarms.
      \item Simple design and relatively low cost compared to other detection methods.
      \item Can withstand harsh environments and are less susceptible to false alarms from non-fire sources.
    \end{enumerate}

    \item Disadvantages:
    \begin{enumerate}
      \item May have a slower response time compared to smoke detectors, as they rely on temperature thresholds being reached.
      \item Limited in their ability to detect fires in the early stages when smoke may not be present.
      \item Cannot differentiate between different types of fires or provide specific location information.
    \end{enumerate}

    \item   Applications:
    \begin{enumerate}
      \item Suitable for environments with high levels of dust, fumes, or steam, such as manufacturing plants, kitchens, or areas with heavy machinery.
    \end{enumerate}
    
  \end{enumerate}
  \item Smoke Detectors:
  \begin{enumerate}
    \item Advantages:
    \begin{enumerate}
      \item Early detection of fires by sensing the presence of smoke particles.
      \item Can provide quick response in detecting smoldering or slow-burning fires.
      \item Available in various types (ionization, photoelectric) to accommodate different fire scenarios.
    \end{enumerate}
    \item Disadvantages:
    \begin{enumerate}
      \item Can be prone to false alarms from non-fire sources, such as cooking smoke or steam.
      \item May not detect fires without the presence of smoke, such as fast-flaming fires.
      \item Smoke detectors can be sensitive to environmental conditions and require regular maintenance.
    \end{enumerate}
    \item Applications:
    \begin{enumerate}
      \item Residential, commercial, and industrial buildings, including bedrooms, hallways, offices, and storage areas.
    \end{enumerate}
    
  \end{enumerate}
  
  \item CO Detectors:
  \begin{enumerate}
    \item Advantages:
    \begin{enumerate}
      \item Specifically designed to detect carbon monoxide, a toxic gas produced by incomplete combustion.
      \item Provide early warning of CO presence, allowing occupants to evacuate and seek fresh air.
      \item Can be integrated with other alarm systems for comprehensive safety measures.
    \end{enumerate}
    \item Disadvantages:
    \begin{enumerate}
      \item Limited to detecting carbon monoxide gas and not other types of fires.
      \item CO detectors may require periodic calibration and maintenance.
    \end{enumerate}
    
    \item Applications:
    \begin{enumerate}
      \item Residential homes, commercial buildings, garages, and areas with fuel-burning appliances like furnaces, stoves, and water heaters.
    \end{enumerate}

  \end{enumerate}
  
  \item Multi-sensor Detectors:
  \begin{enumerate}
    \item Advantages:
    \begin{enumerate}

      \item Combine multiple detection technologies (such as smoke and heat) for improved accuracy and reduced false alarms.
      \item Provide a comprehensive approach to fire detection by analyzing multiple parameters simultaneously.
      \item Can adapt to different fire scenarios and provide more reliable detection.
    \end{enumerate}
    \item Disadvantages:
      \begin{enumerate}
      \item More complex design and higher cost compared to single-sensor detectors.
      \item Regular maintenance and calibration may be required to ensure optimal performance.
      \end{enumerate}
      \item Applications:
      \begin{enumerate}
        \item Areas where accurate and reliable fire detection is critical, such as hospitals, data centers, and museums.
      \end{enumerate}
  
  \end{enumerate}
  
  \item Manual Call Points:
  \begin{enumerate}
    \item Advantages:
    \begin{enumerate}
      \item Enable occupants to manually activate the fire alarm system in emergency situations.
      \item Provide a visible and easily accessible means of initiating an alarm.
      \item Can be strategically placed to ensure quick and direct access.
    \end{enumerate}
    
    \item Disadvantages:
    \begin{enumerate}
      \item Prone to accidental or false activations if not used properly or protected against misuse.
      \item Limited functionality compared to automated detection devices.
    \end{enumerate}

    
    
    \item Applications:
    \begin{enumerate}
      \item Throughout buildings, particularly in exit routes, common areas, and areas prone to fire hazards, to allow individuals to initiate the alarm and prompt evacuation. 
    \end{enumerate}
  \end{enumerate}


\end{enumerate}

\subsubsection*{Smoke Detection System}
\begin{itemize}
  \item \textbf{Ionization Smoke Detectors:} Ionization smoke detectors work by using a small amount of radioactive material to ionize the air within the detector. When smoke particles enter the detector, they disrupt the electrical current, triggering the alarm. Ionization detectors are particularly sensitive to fast-flaming fires that produce small smoke particles.

  \item \textbf{Light Scattering Smoke Detectors}: Light scattering smoke detectors, also known as photoelectric smoke detectors, use a light source and a light sensor to detect smoke. When smoke enters the detection chamber, it scatters the light, causing it to reach the light sensor and trigger the alarm. These detectors are effective at detecting slow-burning, smoldering fires that produce larger smoke particles.
  
  \item \textbf{Light Obscuring Smoke Detectors}: Light obscuring smoke detectors, also referred to as beam detectors, use a transmitter and a receiver that emit and detect a beam of light across an area. When smoke particles obstruct the light beam, either by scattering or absorbing the light, the detector activates the alarm. They are commonly used in large open spaces such as warehouses or atriums.
\end{itemize}


\subsubsection*{Fire alarm system} 
\begin{itemize}
  \item \textbf{Conventional Alarm System}: Divides the building into zones, providing a general indication of the alarm area without specific device location information.

  \item \textbf{Addressable Alarm System}: Each device has a unique address, allowing for specific device location identification when an alarm is triggered.

  \item \textbf{Intelligent Fire Alarm System}: An intelligent fire alarm system incorporates advanced technology to provide enhanced features and functionality compared to conventional systems. It utilizes addressable devices that can communicate with a central control panel, providing detailed information about the exact location of the alarm. Intelligent systems may include additional features such as self-diagnosis, remote monitoring, and integration with other building systems for improved response and management.

  \item \textbf{Wireless Fire Alarm System}: A wireless fire alarm system uses wireless communication technology to connect the fire alarm devices instead of traditional wired connections. Wireless systems offer flexibility in installation as they eliminate the need for extensive wiring. They can be especially useful in retrofitting existing buildings or areas where running wires is challenging or impractical. Wireless systems maintain reliable communication between devices and the control panel, ensuring prompt alarm notifications and efficient fire detection.
\end{itemize}

\end{document}
