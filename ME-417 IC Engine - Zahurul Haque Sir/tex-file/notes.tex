\documentclass{article}
\usepackage[margin=2cm]{geometry}
\usepackage{graphicx}
\usepackage[pages=some]{background}
\usepackage{titling}
\usepackage{tabularx}
\usepackage{tikz}
\usepackage{forest}
\usepackage{float}
\usepackage{subfigure}
\usepackage{amsmath}
\usepackage{amssymb}
\usepackage{multicol}

\forestset{
  my box/.style={
    draw,
    rectangle,
    rounded corners,
    fill=gray!20,
    inner sep=6pt,
    minimum width=3cm % Adjust the width as needed
  }
}


\geometry{a4paper}

\backgroundsetup{
    scale=1,
    angle=0,
    opacity=1,
    contents={%
        \includegraphics[width=\paperwidth,height=\paperheight]{institution_logo.jpg}
    }
}

\newcommand{\subtitle}[1]{
    \posttitle{
        \par\end{center}
        \begin{center}\large#1\end{center}
        \vskip0.5em}
}

\title{ME-463}
\author{Md. Hasibul Islam}
\subtitle{IC ENGINES}

\begin{document}
\begin{titlepage}
    \centering
    
    {\Huge\bfseries\maketitle}
    \textbf{Zahurul Haque Sir} \\
    \vspace{2cm}
    \includegraphics[width=8cm]{institution_logo.jpg}
    \vfill
    \vspace*{2cm}
\end{titlepage}

\tableofcontents
\pagebreak
\section{Lecture 01: Introduction} 
\hfill Date: 04/06/2023

\section{Lecture 02: ENGINE FUELS} 
\hfill Date: 06/06/2023
\begin{itemize}
  \item In future → Diesel + 5-10\% bio-diesel
  \item In future → Petrol + Ethanol
  \item High H/C ratio indicates high value of energy \& heating
  \item Gasoline : 31,850 kJ/L 
  \item Average Human power in 0.2 hp, but an 1500 cc car has a power of 60 kW
  \item Octane number → Petrol Engines 
  \item Cetane number → Diesel Engines 
  \item High compression ratio → more knocking 
\end{itemize}

\subsubsection*{Octane Number}
The octane number is a rating used to measure the performance of gasoline (petrol) in spark-ignition engines. It indicates a fuel's resistance to knocking or detonation, which is the spontaneous combustion of the fuel-air mixture in the engine cylinder, causing a knocking sound. Knocking can lead to engine damage and reduced efficiency. The higher the octane number, the more resistant the fuel is to knocking.\\
Typically, two common octane rating methods are used: Research Octane Number (RON) and Motor Octane Number (MON). RON measures a fuel's performance under mild operating conditions, while MON evaluates it under more severe conditions. The octane number displayed at gas stations usually refers to the average of RON and MON, known as the Anti-Knock Index (AKI) or Pump Octane Number (PON).\\
Higher octane number is preferable. 


\subsubsection*{Cetane Number}
The cetane number is a rating used to measure the ignition quality of diesel fuel. It represents the fuel's ability to ignite quickly and burn efficiently in a compression-ignition (diesel) engine. Similar to the octane number, the cetane number is obtained through laboratory tests. It measures the delay between fuel injection and ignition in a diesel engine. Higher cetane numbers result in shorter ignition delays and more complete combustion.
\begin{itemize}
  \item It indicates the ability of self-ignite of engine.
  \item Higher cetane number may cause firing easily. 
  \item Lower cetane number creates knocking, and even create an explode of 2000°C.
  \item At lower cetane number, fuel will be mixed properly, because of time lagging. 
  \item Neither high cetane number, nor low cetane number is preferable. 
\end{itemize}

\subsubsection*{Why Octane number should be high, but cetane number should be in a specific range?}
While a higher octane number is generally preferable to resist knocking in spark-ignition engines, the ideal cetane number for diesel fuel is not too high or too low. Here's why:
\begin{itemize}
  \item \textbf{Too High Cetane Number}: If the cetane number of diesel fuel is excessively high, it can lead to a phenomenon known as "cetane number related ignition delay." This means the fuel ignites too quickly during the compression stroke, causing a rapid rise in pressure. This can result in increased engine noise, rough combustion, and potentially higher emissions.
  \item \textbf{Too Low Cetane Number}: On the other hand, if the cetane number is too low, the fuel may have a longer ignition delay, causing a delayed start of combustion in the diesel engine. This can lead to difficult cold-starting, rough idling, reduced engine performance, and increased emissions.
\end{itemize}

To achieve optimal performance, diesel fuels are typically formulated to have cetane numbers within a specified range that balances ignition quality, combustion efficiency, and emissions control. The specific range may vary depending on engine design, regional fuel standards, and operating conditions.It's important to note that the cetane number requirements may differ for different diesel engines and applications. For instance, high-performance engines or heavy-duty diesel engines may have different cetane number recommendations compared to standard passenger vehicle diesel engines. By maintaining an appropriate cetane number range for diesel fuel, it ensures proper ignition characteristics, smooth combustion, good cold-starting performance, improved fuel efficiency, and reduced emissions in diesel engines.

\begin{figure}
  \centering
  \includegraphics[width=0.95\textwidth]{img/energy.png}
  \caption{Primary Energy Sources.}
  \label{fig:Primary Energy Sources }
\end{figure}

\subsubsection*{Compression Ratio}
Compression ratio is a fundamental parameter used to describe the internal combustion process in an engine. It represents the ratio of the total cylinder volume when the piston is at the bottom of its stroke (bottom dead center) to the volume when the piston is at the top of its stroke (top dead center). In simpler terms, it quantifies how much the air-fuel mixture is compressed in the engine cylinder.\\

The compression ratio affects engine performance and efficiency in several ways, including its relationship with knocking:
\begin{itemize}
  \item Engine Power: A higher compression ratio generally leads to increased engine power output. This is because a greater compression ratio allows for more efficient combustion, resulting in better utilization of the fuel's energy.
  \item Engine Efficiency: A higher compression ratio can improve engine efficiency by extracting more energy from the fuel-air mixture. This is achieved through increased thermal efficiency, where more of the heat energy released during combustion is converted into useful work.
  \item Knocking: Knocking, also known as detonation, is an undesirable phenomenon where the air-fuel mixture in the cylinder detonates spontaneously before the spark plug ignites it. Knocking causes a knocking sound and can lead to engine damage if it occurs excessively.

\end{itemize}

The compression ratio has a significant impact on the likelihood of knocking. A higher compression ratio increases the cylinder pressure and temperature during compression, making the air-fuel mixture more prone to auto-ignition. If the fuel's octane rating is not high enough to resist knocking under the increased pressure and temperature, knocking can occur. The more compression ratio, the more knocking will happen.

To mitigate knocking, it is essential to use fuels with higher octane numbers in engines with higher compression ratios. Fuels with higher octane ratings have increased resistance to knocking, allowing them to withstand the higher pressures and temperatures associated with higher compression ratios.

\subsubsection*{H//C ratio}
The hydrogen-to-carbon ratio (H/C ratio) is a measure of the relative abundance of hydrogen and carbon atoms in a fuel molecule. It is commonly used to characterize the composition and properties of various fuels. The H/C ratio affects the energy content, combustion efficiency, and emission characteristics of a fuel.

\begin{figure}
  \centering
  \includegraphics[width=0.75\textwidth]{img/hc_ratio.png}
  \caption{H:C atomic ratio of various inorganic hydrocarbon compounds.}
  \label{fig:Different H/C ratio }
\end{figure}

In general, fuels with higher H/C ratios tend to have higher energy content, as hydrogen has a higher energy content per unit mass compared to carbon. Fuels with higher H/C ratios also tend to burn more cleanly, producing fewer carbon dioxide (CO2) emissions and less particulate matter during combustion.


\subsubsection*{Diesel Fuel Specifications}
The terms 1-D, 2-D, and 4-D are commonly used to classify different grades or types of diesel fuel. These classifications are based on the volatility and viscosity characteristics of the fuel, and they are often associated with specific applications. Here's an overview of each type and their uses:

\begin{itemize}
  \item 1-D Diesel Fuel:
    \begin{itemize}
      \item 1-D diesel fuel is a lighter grade of diesel fuel that has a lower viscosity and higher volatility compared to 2-D and 4-D fuels.
      \item It is commonly used in colder climates or during winter months, as it has better cold flow properties and can prevent wax crystal formation at lower temperatures.
      \item 1-D fuel is often referred to as "winter diesel" or "arctic diesel" and is designed to perform well in low-temperature conditions.
      \item It is commonly used in applications such as transportation, agriculture, construction, and mining equipment operating in cold regions.
    \end{itemize}
  \item 2-D Diesel Fuel:
      \begin{itemize}
        \item 2-D diesel fuel is a mid-range grade of diesel fuel that has a moderate viscosity and volatility.
        \item It is suitable for use in a wide range of diesel engines and is the most commonly available type of diesel fuel in most areas.
        \item 2-D fuel is used in various applications, including passenger vehicles, trucks, buses, generators, boats, and industrial machinery.
        \item It is also used in heating systems, as it can be burned in oil furnaces or boilers for space heating.
      \end{itemize}

  \item 4-D Diesel Fuel:
      \begin{itemize}
        \item 4-D diesel fuel is a heavier grade of diesel fuel with a higher viscosity and lower volatility compared to 1-D and 2-D fuels.
        \item It is primarily used in industrial applications and heavy-duty engines that require more robust fuel properties.
        \item 4-D fuel is commonly used in large marine engines, locomotives, power generation, and industrial equipment.
        \item It may also be used in off-road vehicles and machinery, such as construction equipment and agricultural machinery.
      \end{itemize}

\end{itemize}

\subsubsection*{Automotive Fuels}
Automotive fuels are the types of fuels used to power vehicles, specifically designed for use in internal combustion engines found in cars, motorcycles, trucks, and other vehicles. The two primary types of automotive fuels are gasoline and diesel fuel.

\textbf{Gasoline}: Gasoline, also known as petrol, is a volatile fuel primarily used in spark-ignition engines. It is a mixture of hydrocarbons derived from crude oil through refining processes. Gasoline is designed to combust in spark-ignition engines, where a spark from the spark plug ignites the fuel-air mixture, generating power.

\textbf{Diesel Fuel}: Diesel fuel is a heavier and less volatile fuel used in compression-ignition engines, commonly known as diesel engines. It contains higher energy content compared to gasoline and is ignited through compression rather than a spark. Diesel engines compress the air in the cylinder, raising its temperature and allowing diesel fuel to ignite spontaneously upon injection into the cylinder.

Both gasoline and diesel fuel are refined from crude oil, but they have different properties and combustion characteristics due to variations in refining processes. These fuels are distributed through fuel stations or gas stations, where vehicles can be refueled.

Alternative automotive fuels, such as ethanol, biodiesel, natural gas (CNG/LNG), hydrogen, and electric power, are also gaining prominence as alternatives to traditional gasoline and diesel fuels. These alternative fuels aim to reduce emissions, enhance fuel efficiency, and promote environmental sustainability in the transportation sector.

\subsubsection*{Typical Properties of Automotive fuels}
\textbf{Gasoline}:
  \begin{itemize}
    \item Octane Number: Measures the fuel's resistance to knocking or premature combustion.
    \item Vapor Pressure: Indicates the fuel's ability to vaporize at specific temperatures.
    \item Reid Vapor Pressure (RVP): Measures the vapor pressure of gasoline at 100 degrees Fahrenheit (37.8 degrees Celsius).
    \item Ethanol Content: Percentage of ethanol added as an oxygenate in gasoline.
    \item Energy Content: Amount of energy per unit volume (megajoules per liter or British thermal units per gallon).
    \item Density: Mass per unit volume of gasoline.
    \item Research Octane Number (RON): A measure of gasoline's resistance to knocking under controlled conditions.
    \item Motor Octane Number (MON): A measure of gasoline's resistance to knocking under more severe conditions.
  \end{itemize}

\textbf{Diesel Fuel}:
    \begin{itemize}
      \item Cetane Number: Measures the ignition quality of diesel fuel.
      \item Sulfur Content: Amount of sulfur present in diesel fuel, which affects emissions.
      \item Energy Content: Amount of energy per unit volume (megajoules per liter or British thermal units per gallon).
      \item Density: Mass per unit volume of diesel fuel.
      \item Lubricity: Ability of diesel fuel to provide lubrication to fuel system components.
      \item Distillation Range: The temperature range at which different components of diesel fuel vaporize.
      \item Cold Flow Properties: Parameters such as the cloud point, pour point, and cold filter plugging point (CFPP) that determine the fuel's performance in cold temperatures.
    \end{itemize}


  \subsubsection*{Aviation Fuels}
  Aviation fuel, also known as aviation gasoline (Avgas) and jet fuel, is a specialized type of fuel designed for use in aircraft. The properties of aviation fuel are specifically formulated to meet the unique requirements of aviation engines. \textbf{Typically aviation fuels are similar to Kerosine, but sulfer content very low.}

  Here's an overview of aviation fuel and its typical properties:

\textbf{Aviation Gasoline (Avgas)}:

\begin{itemize}
  \item Aviation gasoline is primarily used in piston-engine aircraft.
  \item Octane Rating: Avgas has high octane ratings, typically ranging from 91 to 130, to prevent knocking in high-performance aircraft engines.
  \item Lead Content: Some Avgas formulations may contain lead additives for added octane rating. However, there is a global push to transition to unleaded Avgas to minimize environmental impact.
  \item Density: Avgas has a specific gravity that varies depending on the specific formulation but typically falls between 0.72 and 0.78 kg/L (6.0 to 6.5 lb/gal).
  \item Vapor Pressure: Avgas has controlled vapor pressure to ensure reliable fuel delivery in aircraft systems across a range of altitudes and temperatures.
  \item Flash Point: The flash point of Avgas is typically around -40 to -45 degrees Celsius (-40 to -49 degrees Fahrenheit), indicating its low flammability.
\end{itemize}


\textbf{Jet Fuel}:
\begin{itemize}
  \item Jet fuel is used in gas turbine engines, including turbojets, turbofans, and turboprops.
  \item Jet A and Jet A-1: Jet A and Jet A-1 are the most common types of jet fuel used internationally. They have similar properties and are often interchangeable.
  \item Jet A/A-1 is a kerosene-based fuel with a relatively high flash point, making it less volatile than gasoline.
  \item Density: Jet fuel has a specific gravity around 0.8 kg/L (6.7 lb/gal).
  \item Energy Content: Jet fuel has a high energy content, typically around 35 to 42 megajoules per liter (130,000 to 160,000 British thermal units per gallon).
  \item Freezing Point: Jet fuel has a low freezing point to ensure it remains liquid at low temperatures encountered at high altitudes.
  \item Sulfur Content: International standards, such as Jet A-1, mandate low sulfur content (typically less than 0.3\% mass) to reduce emissions.
\end{itemize}



\newpage

\section{Lecture 3: Combustion stoichiometry}
\hfill{Date: 13/06/2023}

$$\underbrace{C_\alpha H_\beta O_\gamma N_\delta}_{fuel} + \underbrace{a_s \left(O_2 + 3.76 N_2\right)}_{air} \rightarrow n_1CO_2 + n_2H_2O + n_3N2$$ 
here, $a_s \equiv$ stoichiomertric molar air fuel ratio. 1 mol $O_2$ is associated with 3.76 mol $N_2$. By solving, we get,
$$a_s = \alpha + \frac{\beta}{4} - \frac{\gamma}{2}$$
$\left(\frac{A}{F}\right)_s$ = stoichiometric air fuel ratio (mass basis) \\
$\left(\frac{A}{F}\right)_a$ = actual air fuel ratio 
$$\left(\frac{A}{F}\right)_s = \left(\frac{F}{A}\right)_s^{-1} = \frac{28.85\times 4.76 a_s}{12\alpha + \beta + 16 \gamma + 14 \delta}$$
here, 28.85 is the molar mass of air. \\
Most of the combustion engine runs at stoichoimetric condition. 

$\Phi$ = Fuel air equivalence ratio. simply called as equivalence ratio. 

$$\Phi = \lambda^{-1} = \frac{\left(\frac{F}{A}\right)_a}{\left(\frac{F}{A}\right)_s}$$
$$\Phi = \begin{cases}
  < 1 : \text{Lean mixure} \\
  = 1 : \text{Stoichiometric mixure} \\
  > 1 : \text{Rich mixure} 
\end{cases}$$

\subsection*{Heating Value}
Heating value, also known as calorific value or energy content, refers to the amount of heat energy released per unit mass or volume of a fuel during complete combustion. It is a measure of the energy content or potential of a fuel and is typically expressed in units such as joules per kilogram (J/kg) or British thermal units per pound (BTU/lb).

Heating value is an important parameter used to assess the energy efficiency and suitability of different fuels for various applications, such as power generation, heating systems, and industrial processes. It indicates the amount of heat energy that can be obtained by burning a specific fuel completely.

The heating value of a fuel is influenced by its chemical composition, including the amount of carbon, hydrogen, sulfur, and other elements present. Different types of fuels, such as coal, natural gas, gasoline, and biomass, have different heating values due to variations in their chemical properties.

\subsubsection*{Lower Heating Value (LHV)}
LHV (LCV) considers that the water vapor formed during combustion is condensed, and the heat released by condensation is recovered and utilized. This means that LHV accounts for the heat of vaporization of water and assumes that the water in the combustion products is in a liquid state. LHV does not include the heat carried away by the water vapor in the exhaust gases.

\subsubsection*{Lower Heating Value (HHV)}
HHV (HCV), on the other hand, considers that the water vapor formed during combustion remains in a gaseous state and carries away its heat of vaporization. HHV includes the heat carried away by the water vapor in the exhaust gases.

\subsubsection*{For IC engine, which one should use:}
The choice of which value to consider depends on the specific application. In the case of internal combustion engines (IC engines), it is common to use the Lower Heating Value (LHV) because it reflects the actual heat available for useful work in the engine. This is because the water vapor in the exhaust gases typically remains in the gaseous state and does not contribute to the work output of the engine. Using LHV allows for a more accurate estimation of the actual energy content of the fuel that can be utilized by the engine. It accounts for the fact that the water vapor in the exhaust gases does not contribute to the engine's power output and cannot be effectively utilized.

\subsubsection*{Estimate the HHV of $CH_4$ (Methane)}
The higher heating value (HHV) of $CH_4$ (methane) at constant volume can be estimated based on the chemical equation for its complete combustion:

$$CH_4 + 2(O_2 + 3.76N_2) \rightarrow CO_2 + 2H_2O + 7.52N_2$$

The balanced equation shows that for every one mole of methane ($CH_4$) combusted, one mole of carbon dioxide ($CO_2$), two moles of water ($H_2O$), and 7.52 moles of nitrogen ($N_2$) are produced.\\



To estimate the HHV, we need to consider the enthalpy of formation ($\Delta H_f$) values for $CH_4, CO_2, H_2O$, and $N_2$.\\
The enthalpy of formation values at 25°C and 1 atm pressure are:\\
$\Delta H_f(CH_4)$ = -74.8 kJ/mol\\
$\Delta H_f(CO_2)$ = -393.5 kJ/mol\\
$ \Delta H_f(H_2O)$ = -285.8 kJ/mol\\
$\Delta H_f(N_2)$ = 0 kJ/mol\\

Using these values, we can calculate the HHV of CH4 by summing up the enthalpy changes ($\Delta H$) for the products and subtracting the enthalpy changes for the reactants:\\

$$HHV = \Delta H_f(CO_2) + 2\Delta H_f(H_2O) - \Delta H_f(CH_4) - 7.52\Delta H_f(N_2)$$

$$HHV = -393.5 kJ/mol + 2(-285.8 kJ/mol) - (-74.8 kJ/mol) - 7.52(0 kJ/mol)$$

Simplifying the calculation, we have:

$$HHV \thickapprox  -393.5 kJ/mol - 571.6 kJ/mol + 74.8 kJ/mol$$

$$ HHV \thickapprox -890.3 kJ/mol$$

Therefore, the estimated higher heating value (HHV) of methane ($CH_4$) at constant volume is approximately -890.3 kJ/mol.

\subsubsection*{Some Important Points:}
\begin{itemize}
  \item Heating value is calculated for 25°C temperature. 
  \item Constant pressure heating value is more used in boiler and others.
  \item Higher heating value is also called "Gross Heating Value", and Lower Heating Value is called "Net Heating Value". 
  \item Maximum countries use LHV for IC engine. 
  \item For coal \& octane, there is no difference between them for higher heating value. 
  \item $CH_4$ : The hydrogen burning here make a difference of 10\% between HHV \& LHV.  
\end{itemize}

\subsection*{Adiabatic Flame Temperature}
Adiabatic flame temperature refers to the maximum temperature reached during the combustion process of a fuel when no heat is lost to the surroundings (i.e., in an adiabatic and reversible process). It represents the idealized scenario where combustion occurs without any heat transfer to or from the surroundings.\\
If 1 kg fuel is burnt in stoichoimetric and isolated condition, then how much temperature it produces is known as adiabatic flame temperature. 

The adiabatic flame temperature is typically higher than the actual flame temperature observed in practical combustion systems due to heat losses and inefficiencies. The actual peak temperature are less due to - 
\begin{itemize}
  \item Not 100\% burning 
  \item Loss heat from the flame (Radiation)
  \item Endothermic Reaction 
\end{itemize}

\subsubsection*{Effects of Adiabatic Flame Temperature}
If adiabatic flame temperature is higher, engine will run at a higher speed. 

\subsubsection*{Pros and Cons of Higher adiabatic flame temperature}
\begin{enumerate}
  \item Positive Effects:
  \begin{enumerate}
    \item \textbf{Increased Thermal Efficiency}: A higher adiabatic flame temperature can lead to improved thermal efficiency of the engine. It means a greater proportion of the fuel's energy is converted into useful work, resulting in improved engine performance.
    \item \textbf{Increased Power Output}: Higher flame temperature can result in increased pressure and temperature in the combustion chamber, leading to higher expansion ratios and increased power output. This can be advantageous for applications where high power is desired, such as in high-performance engines.
  \end{enumerate}
  
  
  
  
  \item Negative Effects:
  \begin{enumerate}
    \item \textbf{Increased NOx Emissions}: Higher flame temperature can contribute to increased production of nitrogen oxides (NOx) during the combustion process. NOx emissions are associated with environmental concerns and contribute to air pollution. Controlling and reducing NOx emissions becomes crucial for meeting emission regulations.
    \item \textbf{Increased Heat Losses}: A higher flame temperature can increase heat losses through various mechanisms such as increased radiation and convective heat transfer. These losses reduce the actual thermal efficiency of the engine and can impact overall performance.
    \item \textbf{Increased Combustion Temperatures}: Higher flame temperature can result in higher peak combustion temperatures, which may lead to increased formation of harmful pollutants, such as particulate matter (PM) and unburned hydrocarbons. Managing these emissions becomes essential for meeting emission standards and ensuring environmental sustainability.
    \item \textbf{Increased Engine Wear}: Higher temperatures in the combustion chamber can lead to increased thermal stresses on engine components, potentially causing accelerated wear and reduced engine lifespan. This is particularly relevant for components such as piston rings, valves, and cylinder walls.
  \end{enumerate}
  
\end{enumerate}
$\checkmark$ No accurate relation with adiabatic flow temperature and burning velocity is found. \\
$\checkmark$ Engine fuel burning temperature is more than 2500°C, 1500°C temperature is considered as cold enough. \\
$\checkmark$ Mixing $H_2$ will give a higher engine speed. \\
\vspace*{2cm}

\section{Lecture 4: Equilibrium Composition (ISO-Octane at 30 bar)}
\hfill Date: 17/06/2023
\subsection*{Points}
\begin{enumerate}
  \item \textbf{Equivalence Ratio}: Equivalence ratio in an internal combustion engine refers to the ratio of the actual air-fuel mixture to the stoichiometric air-fuel ratio. It determines the richness or leanness of the mixture and affects combustion efficiency, power output, fuel consumption, and emissions. Operating at the stoichiometric ratio provides a balance between efficiency and emissions, but adjustments to the equivalence ratio can be made for optimal performance under different conditions. 
  \item Generally Equivalence ratio is maintained near 1. 
  \item At 500°C octane break down in other compounds like - Methane, Ethane, Butane or others. More than 100 radicals are produced. They further reacts and continue. 
  \item Bigger bond, like - (C-C) bond breaks and bond energy released and heat produced.
  \item H radicals are most critical. The fuel burning will increase in huge amount, if $H_2$ gas is added. 
  \item If fuel percentage is increased, they won't get enough time for fully burning. As a result incomplete burning will occur. Although there will be incomplete burning, there will produce a lot energy. Because of breaking bonds and energy releases. Increase in fuel will not good for the efficiency, but for power, it'll be good. 
  \item If fuel is higher, incomplete combustion will occur and $CO$ (carbon monoxide) like substance will be produced. 
  \item If everything remains same, the higher the RPM, the more power will be produced. 
  \item Fuel burning will be faster in presence of $H_2$ in air fuel mixure. 
  \item To get more power, $H_2$ gas is sucked to the cylinder from the exhaust gas. (As $H_2$ gas is there in exhaust gas) It's known as exhaust gas recirculation. 
  \item At 1750°C temperature, Produces gases in lean or rich condition are  totally different from eath other.
  \item Efficient engine (more than 50\%), their RPM are generally lower (nearly 100 RPM). As, In lower RPM, air fuel mixures will get enough time to burn fully.
  \item The less heat of combustion, the more less the specific heat capacity. 
  \item Maximum Flate temp at $\phi$ = 1.05. But for $\phi$ > 1, heat combustion and heat capacity will decay. 
\end{enumerate}

\vspace*{0.5cm}
\begin{center}
  \begin{tabular}{ccc}
    \hline
     & Air fuel ratio, $\left(\frac{A}{F}\right)_s$ & LHV \\
    \hline
   Methnol & 6.43 & 19.9 \\
    Octane & 15.03 & 44.4 \\
    \hline
  \end{tabular}
  \end{center}

  \subsubsection*{Which one is better: Methanol or Octane?}
  In spite of having more LHV in octane, it has a higher air-to-fuel ratio. That means, 1 portion of fuel will take 15.03 portion of air. As a result, it will occupy more volume in cylinder. On the otherhand, methanol has a lower LHV, also lower air-to-fuel ratio. But, for the same volume of cylinder, methanol will give better performance. As, more methanol can be contained inside the cylinder. \textbf{So, It can not be decided to choose a fuel, only observing the Heating values.}\\
  There are also some benefits of using alcohol as fuel. More compression ratio is better for IC engines. In normal car, knocking will be an issue when compression ratio is higher than 12. But, in case of alcohol, the compression ratio of 18 is enough good. That's why, sports car normally uses alcohol instead of octane for more power. The more the compression ratio, the more the efficiency. 
  
  \subsubsection*{Some Important Points}
  \begin{itemize}
    \item SIT (Self Ignition Temperature) of Diesel is lower than octane. So, diesel engine gives better performance as it needs less temperature to ignite. 
    \item The SIT of methane is high. That's why for ignition in CNG cars, it requires more temperature in spark plug and thus it becomes out of work. 
  \end{itemize}

  \subsubsection*{Combustion efficiency in ICEs}
  $$\eta_c = \frac{H_R (T_o) - H_P (T_o)}{m_f Q_{HV}}$$

  Efficiency will never be 100\%. Because, fuel attached with cylinder wall have less temperature, like - 300-400°C. Whereas, burn temperature in 2500°C. So, Fuel along with wall doesn't burn at ease and requires much time to burn out. \\

  Any fuel can be used in Fuel engine. But, in petrol engine, fuel needs to make vapor first, that's why efficiency is an issue here. 

  \vspace*{0.5cm}
  \section{Lecture 5: Combustion \& Flame}
  \hfill Date: 20/06/2023 \\

  For combustion three parameters are important: \textbf{Power, Efficiency \& Emission}\\


  \textbullet Flame: A rapid exothermic chemical reaction. \\
  \textbullet Flame:  Conventional spark-ignition (SI) flame is premixed unsteady turbulent flame, and the fuel-air mixture through which it propagates is in the
  gaseous state.

  Conventional spark-ignition (SI) flame is premixed unsteady turbulent
  flame, and the fuel-air mixture through which it propagates is in the
  gaseous state.

  Diesel engine (CI) combustion process is predominantly an unsteady
  turbulent diffusion flame, and the fuel is initially in the liquid phase\\

  \textbf{Premixed Flame}: fuel and oxidizer are essentially uniformly mixed
  prior to combustion. It is a rapid, essentially isobaric, exothermic
  reaction of gaseous fuel and oxidizer, and flame propagates as a thin
  zone with speeds of less than a few m/s.\\

  \textbf{Diffusion Flame}: reactants are not premixed and must mix together
  in the same region where reactions take place. It is dominated by the
  mixing of reactants, which can be either laminar or turbulent, and
  reaction takes place at the interface between the fuel and oxidizer

  \subsection*{Auto Ignition \& Self-Ignition Temperature:}
  \textbf{Auto Ignition Temperature:} The auto ignition temperature is the lowest temperature at which the air-fuel mixture can auto-ignite under specific conditions of pressure and composition. It is typically associated with gasoline engines and is the temperature at which knocking or detonation can occur if it is reached before the spark plug fires. The auto ignition temperature of gasoline is typically around 495-535 degrees Celsius (923-995 degrees Fahrenheit).

\textbf{Self Ignition Temperature:} The self-ignition temperature, also known as the ignition point or the ignition temperature, is the minimum temperature at which a fuel will self-ignite without the presence of an external ignition source. It is more commonly associated with diesel engines, where the fuel is injected into the hot, compressed air in the combustion chamber. The self-ignition temperature of diesel fuel is generally higher than that of gasoline, ranging from about 210-260 degrees Celsius (410-500 degrees Fahrenheit).

\subsubsection*{Effect of SIT \& AIT}
In general, for internal combustion engines, both gasoline and diesel, higher auto-ignition and self-ignition temperatures are preferred.\\

For auto-ignition in gasoline engines, a higher temperature threshold is preferred to prevent premature or uncontrolled combustion, such as knocking or detonation. Knocking can lead to engine damage, reduced performance, and increased emissions. Fuels with higher resistance to auto-ignition, indicated by higher auto-ignition temperatures, are desired to ensure proper combustion control and avoid these issues.\\

Similarly, in diesel engines, a higher self-ignition temperature is preferred. Diesel engines rely on the self-ignition of the fuel when injected into the compressed air. A higher self-ignition temperature ensures that ignition occurs at the intended timing and allows for efficient combustion. It also helps in avoiding spontaneous ignition during the compression stroke before the fuel injection, which can cause damage to the engine.\\

As the ignition temperature increases, the ignition time becomes faster and the ignition delay becomes shorter. Higher temperatures promote more efficient combustion by facilitating quicker and more complete fuel combustion. This, in turn, reduces the time it takes for the air-fuel mixture to ignite and lowers the delay between the ignition event and the start of combustion. However, it's important to note that the relationship between ignition temperature and ignition time can be influenced by various factors, including fuel properties, engine design, and operating conditions. 

\subsubsection*{Ignition delay (ID) depends on which factor:}
Factors influencing ignition delay in internal combustion engines include fuel properties (such as cetane number or octane rating),initial temperature, Density, turbulence swirl, compression ratio, air-fuel mixture composition, presence of inert gas, engine operating conditions (speed, load, temperature, pressure), ignition system effectiveness, and combustion chamber design.\\  

\subsection*{Minimum Ignition Energy}


\begin{figure}[h]
	\centering
  
	\subfigure[Ignition Delay]{
	  \includegraphics[width=0.45\linewidth]{img/ignition_delay.jpeg}
	  \label{fig:ID}
	}
	\hfill
	\subfigure[Critical Pressure vs critical temperature]{
	  \includegraphics[width=0.45\linewidth]{img/critical_pv.jpeg}
	  \label{fig:const_vol}
	}
	
	\subfigure[electrode gap]{
	  \includegraphics[width=0.45\linewidth]{img/min_egn_energy.jpeg}
	  \label{fig:const_pressure}
	}
	\hfill
	\subfigure[Minimum Ignition Energy]{
	  \includegraphics[width=0.45\linewidth]{img/min_egn_energy2.jpeg}
	  \label{fig:limited_const_vol_pre}
	}

	\label{fig:combustion_cycle}
  \end{figure}
  
  \begin{itemize}
    \item For lower d, heat loss to electrode is higher  (Heat loss $\propto \frac{1}{d}$)
    \item For higher d, volume increases ($d^3$) . The volume of air fuel mixure will be higher between the eletrodes. 
    \item so, Heat loss will be minimum at a particular distance d. (shown on the figure)
  \end{itemize}

  \subsection*{flammability limit}
  Flammability limits, also known as explosive limits or flammable limits, refer to the range of concentrations of a combustible substance in a mixture with air that can support combustion. These limits define the lower flammable limit (LFL) and upper flammable limit (UFL) of the substance.

\textbf{The lower flammable limit (LFL)} is the minimum concentration of the combustible substance in the air below which there is insufficient fuel for combustion to occur. At concentrations below the LFL, the mixture is too lean to sustain a flame or ignition.

\textbf{The upper flammable limit (UFL)} is the maximum concentration of the combustible substance in the air above which there is too much fuel for combustion to occur. At concentrations above the UFL, the mixture is too rich to sustain a flame or ignition.

Within the flammability limits, the mixture is within the range where combustion can occur if an ignition source is present. This range is often referred to as the flammable range.

The Lower Flammable Limit (LFL) and Upper Flammable Limit (UFL) can depend on temperature. Changes in temperature can affect the flammability limits of a substance. In general, as the temperature increases, the flammability limits tend to widen.

Flame speed significantly increase by turbulance. 

\subsubsection*{How Flame works:}
Nearest molecule of flame reaches the SIT, then burn and increase the furthur temperature of next molecules and propagate to the forwards. 

\begin{figure}[h]
  \begin{center}
    \includegraphics[width=0.95\linewidth]{img/flame_graph.jpeg}
  \end{center}
\end{figure}


\section{Lecture 6: Laminar Flame Propagation }
\hfill Date: 23/06/2023

\subsection*{Laminar Burning Velocity, $S_L$}
It is an intrinsic property of a fuel-air mixure. It is defines as the velocity, relative to \& normal to the flame front, with which unburned gas moves into the front \& is transformed to products under laminar flow conditions. \\
\textbullet if we sit on the flame co-ordinate, how Unburn mixure along with flame co-ordinate exit, the such velocity in perpendicular  way is called the laminar burning velocity.  
$$S_L = \frac{1}{\rho_u A_f} \frac{dm_b}{dt}$$
$$S_S = S_L + u_g$$ 
here, $dm_b/dt$ is mass burning rate and $A_f$ is flame front surface area \\
$S_S$ = flame speed; space velocity of flame front normal to itself. it is not a unique property of combustible fuel air premixure. \\
$u_g$ = gas expansion velocity \& is a punction of $\rho_u$ and $\rho_b$ \\
u (subscript) = unburned mixuyre \\
0 (subscript) = datum pressure and temperature \\

$$S_L= f(fuel, T, P, \Phi) \cong S_{L,0} \left(\frac{T}{T_0}\right)^\alpha \left(\frac{P}{P_0}\right)^\beta$$

\begin{figure*}[h]
  \begin{center}
    \includegraphics[width=0.9\linewidth]{img/laminar_flame.jpeg}
  \end{center}
\end{figure*}

\subsection*{Turbulent Flame Propagation}
$$S_T = f\left(fuel, T, P, \Phi, turbulence\right)$$
$$\frac{S_T}{S_L} \cong f(turbulence) $$

If turbulence is higher, basically burn surface area will increase. Hene, burning velocity will also increase. \\
If turbulence is too high, then burn surface area will be separated from each other, resulting decrease in buring velocity. 


\begin{figure}[h]
  \centering
  \subfigure[Effect on Equivalence ratio]{\includegraphics[width=0.30\linewidth]{img/sl-equival_ratio.jpeg}}
  \hfill
  \subfigure[Effect on unburned mixure temperature]{\includegraphics[width=0.62\linewidth]{img/sl-temp.jpeg}}
  \vspace{0.5cm}
  \subfigure[Effect on unburned mixure pressure]{\includegraphics[width=0.45\linewidth]{img/sl-press.jpeg}}
  \hfill
  \subfigure[Effect of turbulance]{\includegraphics[width=0.45\linewidth]{img/sl-turbu.jpeg}}
  \caption{Effects of Laminar flame velocity.}
  \label{fig:subfigures}
\end{figure}



\subsection*{Effects on Laminar Flame Propagation}
The equivalence ratio, unburned mixture temperature, and unburned mixture pressure have significant effects on laminar flame propagation in internal combustion engines. Here's a breakdown of how each parameter influences the process:
  
  \begin{itemize}
  \item Equivalence Ratio:
  \begin{enumerate}
    \item The equivalence ratio ($\Phi$) is the ratio of the actual fuel-to-air ratio to the stoichiometric fuel-to-air ratio required for complete combustion. It represents the relative richness or leanness of the air-fuel mixture. Higher equivalence ratios ($\Phi > 1$) indicate a rich mixture with excess fuel, while lower equivalence ratios ($\Phi < 1$) represent a lean mixture with excess air.
    \item  Effect on Flame Propagation:\\
    \textbf{Rich Mixtures ($\Phi > 1$):} In rich mixtures, the increased fuel concentration leads to a higher fuel availability for combustion. This can result in faster flame propagation due to the greater availability of reactive fuel molecules.\\
    \textbf{Lean Mixtures ($\Phi < 1$):} Lean mixtures have a higher proportion of air, which can result in slower flame propagation. Lean mixtures may require longer ignition times and may exhibit slower burning rates due to the limited fuel availability.
  \end{enumerate}
  
  
  
  
  \item Unburned Mixture Temperature:
    \begin{enumerate}
      \item The unburned mixture temperature ($T_{unburned}$) refers to the temperature of the air-fuel mixture prior to combustion. It can vary depending on factors such as engine operating conditions, intake air temperature, and compression ratio.
      \item Effect on Flame Propagation:\\
      $\alpha$ is +ve. Increase in temperature increases chemical reaction rates and dissociations. So, more active radicals are produced to enhance flame
      propagation.
    \end{enumerate}
   
  
  \item Unburned Mixture Pressure:
  \begin{enumerate}
    \item The unburned mixture pressure ($P_{unburned}$) represents the pressure of the air-fuel mixture before combustion. It is influenced by factors such as intake air pressure, compression ratio, and engine load.
    \item Effect on Flame Propagation:\\
    $\beta$ is either zero or negative. Increased pressure increases flame temperature
    because of less dissociation, and less dissociation means less active radicals are available to diffuse upstream to enhance flame propagation
  \end{enumerate}
   
  
\end{itemize}

\section{Lecture 7: Combustion in SI engine }
\hfill Date: 04/07/2023

\subsection*{Imporant Points}
\begin{itemize}
  \item In SI engine combustion, For generating active radicals more than 1000°C temperature is required. But, at the cylinder wall of combustion the temperature are nearly 300/400°C. As a result, Flame is quenched. 
  \item Flame is created after a certain distance from the spark plug due to turbulance. 
  \item As the process is isentropic compression, the temperature will rise with the increase of pressure. That means, both will rise in compression process. 
  \item For every cycle, the initial temperature, initial pressure, and other states are completely different from each other. So, graph will be different in every case. 
  \item Density of unburned mixures are 4 times higher than the density of burned mixures. 
\end{itemize}
\vspace*{1cm}
\textbf{Follow The Slide!!!} 

\vspace*{2cm}
\section{Lecture 08: Essential Features of Combustion Process} 
\hfill Date: 08/07/2023


\textbf{Follow The Slide!!}

\subsection*{Important Points}
\begin{itemize}
  \item The higher the compression ratio, the lower the ignition delay 
  \item The higher the equivalence ratio, the higher the ignition delay. At the first, slowly fuel is injected by spray. 
  \item For Compression ratio, better efficiency found.   
  \item Most efficient engines: i) stratified charge engine, ii) lean-burn engine 
\end{itemize}
\vspace*{1cm}

\section{Lecture 09: Abnormal Knock} 
\hfill Date: 04/06/2023

\subsection*{Important points}
\begin{itemize}
  \item Those who decrese the time delay, also decrease the knocking 
  \item Non reactive composition, increase the SIT. 
  \item To decrease the travel time inside cylinder, bore and stroke dimension are taken as same. So, travel time and knocking is decreased. 
  \item Octane rating : Octane rating is a measure of a fuel's resistance to "knocking" or "pinging" during combustion in an internal combustion engine. Knocking refers to the undesirable phenomenon where the air-fuel mixture in the engine's cylinder explodes prematurely and irregularly, causing a knocking sound and potentially damaging the engine.
  \item The octane rating of a fuel indicates its ability to resist knocking. The higher the octane rating, the more resistant the fuel is to knocking. Fuels with higher octane ratings are commonly used in high-performance engines or engines with high compression ratios, as these engines are more prone to knocking.
  \item The octane rating is typically displayed on fuel pumps and is represented by a number, such as 87, 91, or 95. In most countries, the octane rating displayed is the Research Octane Number (RON), which measures the fuel's performance under mild operating conditions. Some countries also use a different octane rating called the Motor Octane Number (MON), which reflects performance under more severe operating conditions.
\end{itemize}

\subsection*{Essential Features of CI engines}
\begin{enumerate}
  \item Compression ratio is high, so can reach higher temperature and pressure at ease. But at that temperature in SI engine, knocking is produced. 
  \item In SI engine, Physical delay is zero. Because, fuel enters after mixing. But in CI engine, both physical and chemical delay are present. 
  \item CI engines are often called slow-speed engines because they are designed to operate at lower rotational speeds compared to SI engines. This is primarily due to the longer combustion process in CI engines, their high torque output at lower RPMs, and the need for sturdier components to withstand high compression pressures. Due to physical mixing, so it takes longer time. While not all CI engines are slow-speed engines, their design characteristics generally make them better suited for lower-speed operation. 
  \item In CI engines, fast evaporation will occur, then mixing will happen. After having a stoichiomertric mixure, some portion will start burning and explode. so, temperature \& pressure will rise faster. As a result, those mixing are not burnt yet, will start burning and exploding. 
  \item Knocking can not be removed fully. Because, there will be some mixing delay and ignition time delay. 
  \item In case of CI engine, those fuels are non-reactive will increase the knocking. whereas, in SI engine, non-reactive fuels decreased the knocking. 
  \item Non-reactive fuels, or fuels with low cetane numbers, can increase the likelihood of knocking in CI engines. Cetane number is a measure of a fuel's ignition quality in CI engines, similar to the octane rating in spark-ignition (SI) engines. When a non-reactive or low cetane fuel is used in a CI engine, it tends to have a longer ignition delay. Ignition delay refers to the time between fuel injection and the start of combustion. With a longer ignition delay, the fuel-air mixture has more time to mix and react in the combustion chamber. This extended time can lead to uneven or delayed ignition, causing knocking. Additionally, non-reactive fuels may have lower heat release rates compared to more reactive fuels. A slower heat release rate means that the fuel takes longer to burn completely, increasing the likelihood of incomplete combustion and knocking. To mitigate knocking in CI engines, it is important to use fuels with higher cetane numbers, which indicates better ignition quality. Fuels with higher cetane numbers have shorter ignition delays and more rapid combustion, reducing the chances of knocking and promoting smoother engine operation.
  \item SI engine fuels are very bad as a CI engine fuel and vice-versa.
  \item If the evaporation time decreases, theoretically, the physical delay will also decrease. But, here surface tension and viscosity are important. First need to atomize the fuel and then evaporation will occur. If evaporation takes place earlier, that can not reach to the last part of the cylinder. These types of problem will occure at lower surface tension and viscosity. That's why, neither too high evaporation time, nor too low evaporation time is good. We have to find out the optimum condition. 
  \item CI engine processes are complicated!
\end{enumerate}


\section{Lecture 10: Diesel Combustion System} 
\hfill Date: 15/07/2023

Two categories of diesel combustion system:
\begin{enumerate}
  \item Direct Inject (DI) Engine: used in large diesel engine
  \item Indirect Inject (IDI) Engine: used in small diesel engine \& mostly used. 
\end{enumerate}

\subsection*{Important Points:}
\begin{itemize}
  \item In diesel engine, rich, lean, and stoichiometric mixures exists at the same time. That's why, here complexity is more. In rich portion there is a lack of oxygen, hence $C, CO$ and others toxic gas is produced. 
  \item Those parameters increase the ignition delay, those are responsible for knocking. 
  \item In winter season knocking is more in diesel engine. In winter, diesel fuel thickens and atomizes less, leading to incomplete combustion and increased knocking. Cold air intake densifies and raises the compression ratio, further promoting knocking. The colder temperature delays fuel ignition, causing a longer ignition delay period. Issues with glow plugs can also contribute to knocking. To reduce knocking, use winter-grade diesel fuel, employ an engine block heater or cold weather fuel additive, and ensure glow plugs are functioning correctly. Regular maintenance is important for preventing knocking in cold weather. 
  \item If engine speed is increased, knocking will also be increased. As, Turbulence increased, in that sense, knocking should decrease, but in real life, opposite thing happens.  Because, Engine speed doesn't increase the evaporation rate, that is all the same. So, more fuel accumulation will occur. Hence, Knocking will increase.   
\end{itemize}

\subsection*{Compare CI and SI engine:}
\textbf{Fuel Ignition:} In SI engines, knock is caused by the uncontrolled ignition of the air-fuel mixture due to end-gas autoignition. On the other hand, in CI engines, the fuel is ignited by the heat of compressed air, without the need for a spark plug.\\

\textbf{Timing:} In SI engines, the spark plug timing determines the ignition of the air-fuel mixture. Advancing the timing excessively can lead to knock. In contrast, CI engines rely on the compression ratio to ignite the fuel, and knock is more likely to occur if the compression ratio is too high.\\

\textbf{Air-Fuel Ratio:} In SI engines, a lean air-fuel mixture is more prone to knock. However, in CI engines, a rich air-fuel mixture can contribute to knock.\\

\textbf{Flame Propagation:} In SI engines, the flame front propagates from the spark plug throughout the combustion chamber. In CI engines, fuel is injected into the combustion chamber, and multiple flame fronts are formed simultaneously.\\ 

\section{Lecture 11: Gas Turbine Combustion} 
\hfill Date: 16/07/2023

Combustion in gas turbine: 
\begin{enumerate}
  \item Recirculation zone 
  \item Burning zone
  \item Dilution zone 
\end{enumerate}

$\checkmark $ Crank-shaft 720° rotation $\rightarrow$ cam shaft 360° rotation  

\textbf{See The Slide!!}

\section{Lecture 12: Air Cleaning \& Silencing} 
\hfill Date: 22/07/2023


\subsection*{Intake manifold if not installed, may seem like more air \& power will be produced. But, it's not true. why?}
Not installing the intake manifold disrupts proper air distribution, optimized airflow, and sensor integration, leading to reduced engine performance, power, and efficiency. Modern engines are finely tuned to work with specific intake systems, so omitting the manifold can cause various issues.

\begin{multicols}{2}
  \subsection*{Important Points:}
  \begin{itemize}
    \item At inlet valve of engine, the velocity of 60-90 m/s 
    \item For air cleaning: air cleaner/filter/silencer is necessary. Also flame arrester for preventing flame generation through intake valve. 
    \item Oil wetted mash is very useful. An oil-wetted mesh air filter is a type of air filtration system used in some older internal combustion engines. It consists of a metal or wire mesh coated with oil. The filter traps dust and dirt as air passes through, preventing contaminants from entering the engine cylinders. 
    \item Oil bath \& Mesh: air is passed through oil (i.e.: irrigation engine). Dust particle can not cross through the oil, hence can not harm engine, but oil will need to be replaced for removing dust.
    \item Cyclone \& fibre: Big engine (centrifugal action removes the dust particles outward)
    \item Helmboltz resonator: Can create 180° resonant sound to mitigate the incoming sound. As wave with phase difference of 180° will cancel each other. 
    \item Exhaust manifold: Cast iron can thermally expand very quickly, so it is helpful for constant start/stop engines.
    \item Exhaust silencer: When passing through multiple layer, pressure drops and so sound heard.
    \item Acoustic lining can prevent small sound.
    \item Exhaust valve are smaller than inlet valve, to maintain higher gas flow velocity, dissipate heat more effectively, reduce weight and inertia, and accommodate space constraints. This design optimizes engine performance, emissions, and fuel efficiency. 
    \item Other reasons, from atmosphere to inlet, the velocity push is not much. But from exhaust (5 atm) to air (1 atm), high velocity push occurs.
    \item Inlet valve subjected to cold temperature, but exhaust valve subjected to hot temperature. It is hard to cool hot temperature, so smaller valve is used. 
    \item If we open the valve at TDC, it wil time some time to open and fill the mixture. So, it will consume some volume before mixing. That's why, valve is opened before TDC. 
    \item When outward flow is closed off, then there exist pressure in the intake pipe, which is very high. So, rather wbile compression strokes occur, the inlet valve is kept open. so pressure from pipe is also felt until the inside pressure is higher than pipe. Important for high speed operatio. 
    \item Ramming effect: The higher the speed, the more pressure. Low speed, Less ramming effect.
    \item Blowdown (exhaust process): In lieu of losinng some pressure opening exhaust valve bit early ensures that inside valve pressure is atmospheric. So, easier to exhaust. 
    \item Exhaust and intake, both are open at the same time. Vaccum created by inertia affect will help for exhaust gas to move faster. so efficiency will increase.

  \end{itemize}
\end{multicols}

\section{Lecture 13: Exhaust valve opening, closing \& Overlapping} 
\hfill Date: 25/07/2023

\begin{multicols}{2}
  \subsection*{Points}
  \begin{itemize}
    \item closing valve $\rightarrow$ ramming effect $\rightarrow$ volumentric efficiency rises 
    \item Ramming effect: The "ramming effect" in an internal combustion engine refers to the phenomenon where the intake air pressure increases as the engine's vehicle moves at higher speeds. This increase in air pressure is caused by the forward motion of the vehicle forcing air into the engine's intake system at a higher rate than would be achieved at rest or lower speeds. As a result, the engine can receive a greater volume of air during each intake stroke, leading to improved engine performance and efficiency at higher speeds. This effect is particularly relevant for engines with naturally aspirated (non-turbocharged) intake systems.
    \item Because of suction effect, a vaccum created but low speed has less impact upon ramming effect 
    \item VVT (variable valve timing) : a) varies cam to crank phasing, b) varies valve timing, c) both 
    \item VTEC (variable valve timing and lift electronic control): control via soleniod push.
    \item Air charging: natural aspiration (all SI engine) relies on pressure difference. 
    \item Pressure wave tuning: Pressure wave tuning is a technique used to optimize intake and exhaust systems in internal combustion engines. By tuning the length and shape of the pipes, pressure waves are timed to enhance gas flow, improving engine performance and efficiency. It is commonly used in high-performance engines to maximize power output. 
    \item Force inductor: forced via higher pressure than air. When pressure increased (increasing compression ratio), then knocking will occur. 
    \item But in CI engine, only air is sucked, so no knocking.
    \item So turbocharged, supercharged for SI engine is bad, for CI engine is good. 
    \item Inertia Ramcharging: Inertia ramcharging is a technique that utilizes the kinetic energy of moving air/fuel mixture to improve intake system efficiency, increasing engine power and performance.
    \item Why CI engine part load efficiency is excellent, and SI engine part load efficiency is poor? In SI engine, this is done by regulation air intake via throttling valve. But in CI engine, this does not need to be controlled.
    \item when intake pipe is very thin, high velocity is achieved. But very high friction occurs.
    \item Pressure wave tuning: The pressure wave travels to end of exhaust manifold, where the density is higher at atmosphere. So, reflection will occur for a part of the shockwave and returns to the intake valve. As such: at last moment, it provides a push of air inside. This is the pressure charging
    \item Mechanical Supercharging: Compressor (belt) taking air from atmosphere and compressing it with help of mechanical shaft of engine.
    \item It only take 2-3\% of engine so not much work loss. Rather more output increases. 
    \item Supercharger are basically blowers (air blowers) pushes air. 
  \end{itemize}
\end{multicols}

\section{Lecture 14} 
\hfill Date: 05/08/2023

\subsection*{Important Points}
\begin{itemize}
  \item CI : turbocharge and supercharge can be used 
  \item SI : Turbo or super can be not be used. Knocking will arise.
  \item With turbo/super, in order to increase the efficiency, intercooler is used. 
  \item In engine, moisture is very important factor.
  \item 3-4\% moisture can be there.
\end{itemize}

\section{Lecture 15}
\hfill Date: 08/08/2023

\textbf{Missed The Class!!!}

\end{document}
