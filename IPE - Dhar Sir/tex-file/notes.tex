\documentclass{article}
\usepackage[margin=2cm]{geometry}
\usepackage{graphicx}
\usepackage[pages=some]{background}
\usepackage{titling}
\usepackage{tabularx}
\usepackage{tikz}
\usepackage{enumitem}

\geometry{a4paper}

\backgroundsetup{
    scale=1,
    angle=0,
    opacity=1,
    contents={%
        \includegraphics[width=\paperwidth,height=\paperheight]{institution_logo.jpg}
    }
}

\newcommand{\subtitle}[1]{
    \posttitle{
        \par\end{center}
        \begin{center}\large#1\end{center}
        \vskip0.5em}
}

\title{IPE-431}
\author{Md. Hasibul Islam}
\subtitle{MACHINE TOOLS}

\begin{document}
\begin{titlepage}
    \centering
    
    {\Huge\bfseries\maketitle}
    \textbf{Nikhil Ranjan Dhar Sir} \\
    \vspace{2cm}
    \includegraphics[width=8cm]{institution_logo.jpg}
    \vfill
    \vspace*{2cm}
\end{titlepage}

\tableofcontents
\pagebreak
\section{Lecture 01: Introduction to Machine Tools} 
\hfill Date: 06/06/2023

\begin{itemize}  
  \setlist[itemize]{label=\Roman*.} 
  \item \textbf{Machine}: A machine is a mechanical or electrical device that is designed to perform specific tasks or operations. Machines are typically composed of various components and mechanisms that work together to achieve a desired outcome. They can be powered by electricity, steam, hydraulic systems, or other sources of energy. Examples of machines include automobiles, computers, washing machines, and assembly line robots.

  \item \textbf{Tool}: A tool is a handheld device or instrument used to perform a specific task or job. Tools are often manually operated and can be powered by hand, electricity, or other means. They are designed to enhance human capabilities and make tasks easier or more efficient. Examples of tools include hammers, screwdrivers, wrenches, drills, and saws.
  
  \item \textbf{Cutting Tool}: A cutting tool is a specific type of tool used to remove material from a workpiece through a cutting or shearing action. Cutting tools are typically designed with sharp edges or blades to slice, shape, or separate materials. They are commonly used in manufacturing, woodworking, metalworking, and other industries. Examples of cutting tools include knives, scissors, drills, milling cutters, lathe tools, and bandsaw blades.
  
  \item \textbf{Machine Tool} : A machine tool is a type of machine that is specifically designed for machining operations, such as shaping, cutting, drilling, or grinding materials. Machine tools are typically powered by electricity or other sources of energy and are used to produce or modify parts or components with high precision. They are essential in the manufacturing industry. Examples of machine tools include lathes, milling machines, drilling machines, grinding machines, and CNC (Computer Numerical Control) machines.
\end{itemize}

\subsubsection*{Charateristics of Machine Tools}
\begin{enumerate}[label=\roman*.]
  \item Precision: Machine tools are designed to perform operations with high precision and accuracy. They are capable of producing parts with tight tolerances, ensuring consistent quality and fit.

\item Repeatability: Machine tools can repeatedly produce identical parts or components with consistent dimensions and features. This is crucial in manufacturing processes where interchangeability and uniformity are required.

\item Rigidity and Stability: Machine tools are built to be rigid and stable, minimizing vibrations and deflections during operation. This ensures the accuracy of machining operations and enhances the tool's ability to handle high cutting forces.

\item Power and Control: Machine tools are typically powered by electrical motors, hydraulic systems, or other sources of energy. They incorporate control systems that enable precise control over the cutting tools, workpiece movement, and machining parameters.

\item Versatility: Machine tools are versatile in their ability to perform a wide range of machining operations. They can be equipped with different tooling and accessories to accommodate various cutting, shaping, drilling, or grinding tasks.

\item Automation Capability: Many modern machine tools are equipped with automation features, such as computer numerical control (CNC) systems. These systems allow for automated tool changes, programmable operations, and integration with other manufacturing processes.

\item Durability: Machine tools are designed to withstand heavy usage and harsh machining conditions. They are typically constructed from sturdy materials and undergo rigorous testing to ensure durability and reliability.

\item Specialization: Machine tools are often designed for specific types of machining operations, such as turning, milling, drilling, grinding, or cutting. Each type of machine tool is optimized for its specific task, allowing for efficient and effective machining.
\end{enumerate}

% \begin{enumerate}[label=\roman*.]
%   \item First item
%   \item Second item
%   \item Third item
% \end{enumerate}

\subsubsection*{Major Category of Manufacturing or Industry}
Different processes or techniques commonly used in manufacturing and related industries:
\begin{enumerate}[label=\alph*.]
  \item \textbf{Forming}: Forming processes involve changing the shape, size, or structure of a material without removing any material. These processes typically rely on applied force or pressure to achieve the desired shape. Forming techniques can include casting, forging, rolling, bending, extrusion, and stamping. These methods are used to create complex shapes, structures, or components from materials such as metals, plastics, and ceramics.
  \item \textbf{Joining}: Joining processes are used to combine two or more separate components or materials into a single unit. These processes create a permanent bond or connection between the parts, allowing them to function as a single entity. Common joining techniques include welding, soldering, brazing, adhesive bonding, mechanical fastening (such as screws or rivets), and fusion processes (like fusion welding or fusion bonding). Joining enables the construction of larger structures, assembly of parts, and the creation of complex systems.
  \item \textbf{Removal}: Removal processes, also known as machining processes, involve removing material from a workpiece to create the desired shape, size, or surface finish. These processes typically employ cutting tools to remove excess material, and they are commonly used for shaping and finishing operations. Examples of removal processes include cutting, turning, milling, drilling, grinding, and abrasive machining. Removal processes are widely utilized in industries such as metalworking, woodworking, and manufacturing.
  \item \textbf{Regenerative}: The term "regenerative" is not a standard category in manufacturing processes. It seems to be used here in the context of sustainability and recycling. In this context, regenerative processes would refer to techniques that involve the reuse, recycling, or regeneration of materials or energy. These processes aim to minimize waste, reduce environmental impact, and promote sustainability. Examples of regenerative processes can include recycling of materials, energy recovery from waste, use of renewable energy sources, and eco-friendly manufacturing practices.
\end{enumerate}









\newpage

\section{Lecture 2: Topic}
\subsection*{Date: DD/MM/YYYY}

Content of the lecture goes here.

\end{document}
