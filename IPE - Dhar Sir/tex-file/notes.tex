\documentclass{article}
\usepackage[margin=2cm]{geometry}
\usepackage{graphicx}
\usepackage[pages=some]{background}
\usepackage{titling}
\usepackage{tabularx}
\usepackage{tikz}
\usepackage{enumitem}

\geometry{a4paper}

\backgroundsetup{
    scale=1,
    angle=0,
    opacity=1,
    contents={%
        \includegraphics[width=\paperwidth,height=\paperheight]{institution_logo.jpg}
    }
}

\newcommand{\subtitle}[1]{
    \posttitle{
        \par\end{center}
        \begin{center}\large#1\end{center}
        \vskip0.5em}
}

\title{IPE-431}
\author{Md. Hasibul Islam}
\subtitle{MACHINE TOOLS}

\begin{document}
\begin{titlepage}
    \centering
    
    {\Huge\bfseries\maketitle}
    \textbf{Nikhil Ranjan Dhar Sir} \\
    \vspace{2cm}
    \includegraphics[width=8cm]{institution_logo.jpg}
    \vfill
    \vspace*{2cm}
\end{titlepage}

\tableofcontents
\pagebreak
\section{Lecture 01: Introduction to Machine Tools} 
\hfill Date: 06/06/2023

\begin{itemize}  
  \setlist[itemize]{label=\Roman*.} 
  \item \textbf{Machine}: A machine is a mechanical or electrical device that is designed to perform specific tasks or operations. Machines are typically composed of various components and mechanisms that work together to achieve a desired outcome. They can be powered by electricity, steam, hydraulic systems, or other sources of energy. Examples of machines include automobiles, computers, washing machines, and assembly line robots.

  \item \textbf{Tool}: A tool is a handheld device or instrument used to perform a specific task or job. Tools are often manually operated and can be powered by hand, electricity, or other means. They are designed to enhance human capabilities and make tasks easier or more efficient. Examples of tools include hammers, screwdrivers, wrenches, drills, and saws.
  
  \item \textbf{Cutting Tool}: A cutting tool is a specific type of tool used to remove material from a workpiece through a cutting or shearing action. Cutting tools are typically designed with sharp edges or blades to slice, shape, or separate materials. They are commonly used in manufacturing, woodworking, metalworking, and other industries. Examples of cutting tools include knives, scissors, drills, milling cutters, lathe tools, and bandsaw blades.
  
  \item \textbf{Machine Tool} : A machine tool is a type of machine that is specifically designed for machining operations, such as shaping, cutting, drilling, or grinding materials. Machine tools are typically powered by electricity or other sources of energy and are used to produce or modify parts or components with high precision. They are essential in the manufacturing industry. Examples of machine tools include lathes, milling machines, drilling machines, grinding machines, and CNC (Computer Numerical Control) machines.
\end{itemize}

\subsubsection*{Charateristics of Machine Tools}
\begin{enumerate}[label=\roman*.]
  \item Precision: Machine tools are designed to perform operations with high precision and accuracy. They are capable of producing parts with tight tolerances, ensuring consistent quality and fit.

\item Repeatability: Machine tools can repeatedly produce identical parts or components with consistent dimensions and features. This is crucial in manufacturing processes where interchangeability and uniformity are required.

\item Rigidity and Stability: Machine tools are built to be rigid and stable, minimizing vibrations and deflections during operation. This ensures the accuracy of machining operations and enhances the tool's ability to handle high cutting forces.

\item Power and Control: Machine tools are typically powered by electrical motors, hydraulic systems, or other sources of energy. They incorporate control systems that enable precise control over the cutting tools, workpiece movement, and machining parameters.

\item Versatility: Machine tools are versatile in their ability to perform a wide range of machining operations. They can be equipped with different tooling and accessories to accommodate various cutting, shaping, drilling, or grinding tasks.

\item Automation Capability: Many modern machine tools are equipped with automation features, such as computer numerical control (CNC) systems. These systems allow for automated tool changes, programmable operations, and integration with other manufacturing processes.

\item Durability: Machine tools are designed to withstand heavy usage and harsh machining conditions. They are typically constructed from sturdy materials and undergo rigorous testing to ensure durability and reliability.

\item Specialization: Machine tools are often designed for specific types of machining operations, such as turning, milling, drilling, grinding, or cutting. Each type of machine tool is optimized for its specific task, allowing for efficient and effective machining.
\end{enumerate}

% \begin{enumerate}[label=\roman*.]
%   \item First item
%   \item Second item
%   \item Third item
% \end{enumerate}

\subsubsection*{Major Category of Manufacturing or Industry}
Different processes or techniques commonly used in manufacturing and related industries:
\begin{enumerate}[label=\alph*.]
  \item \textbf{Forming}: Forming processes involve changing the shape, size, or structure of a material without removing any material. These processes typically rely on applied force or pressure to achieve the desired shape. Forming techniques can include casting, forging, rolling, bending, extrusion, and stamping. These methods are used to create complex shapes, structures, or components from materials such as metals, plastics, and ceramics.
  \item \textbf{Joining}: Joining processes are used to combine two or more separate components or materials into a single unit. These processes create a permanent bond or connection between the parts, allowing them to function as a single entity. Common joining techniques include welding, soldering, brazing, adhesive bonding, mechanical fastening (such as screws or rivets), and fusion processes (like fusion welding or fusion bonding). Joining enables the construction of larger structures, assembly of parts, and the creation of complex systems.
  \item \textbf{Removal}: Removal processes, also known as machining processes, involve removing material from a workpiece to create the desired shape, size, or surface finish. These processes typically employ cutting tools to remove excess material, and they are commonly used for shaping and finishing operations. Examples of removal processes include cutting, turning, milling, drilling, grinding, and abrasive machining. Removal processes are widely utilized in industries such as metalworking, woodworking, and manufacturing.
  \item \textbf{Regenerative}: The term "regenerative" is not a standard category in manufacturing processes. It seems to be used here in the context of sustainability and recycling. In this context, regenerative processes would refer to techniques that involve the reuse, recycling, or regeneration of materials or energy. These processes aim to minimize waste, reduce environmental impact, and promote sustainability. Examples of regenerative processes can include recycling of materials, energy recovery from waste, use of renewable energy sources, and eco-friendly manufacturing practices.
\end{enumerate}









\newpage

\section{Lecture 2: Components \& Functions of Machine tools}
\hfill Date: 13/06/2023

[FOLLOW SLIDE NO. 01]

\subsection*{Links \& Joints}
\begin{itemize}
  \item Rigidity: Increasing the number of links or joints typically reduces the overall rigidity of the system. This is because additional connections introduce more degrees of freedom, allowing for increased flexibility and potential for deformation. The system becomes more prone to deflection and less resistant to external forces.
  \item Flexibility: Increasing the number of links or joints generally enhances the overall flexibility of the system. With more connections and degrees of freedom, the system gains the ability to undergo greater deformation or movement in response to external forces. This increased flexibility can be advantageous in certain applications that require compliance or adaptability to complex environments.
  \item Accuracy: The impact on accuracy with an increase in the number of links or joints is less straightforward and depends on various factors. If the additional links and joints are not well-designed or manufactured, they can introduce more sources of error, potentially reducing accuracy. However, if the added components are carefully integrated and controlled, they can enhance accuracy by providing more control points and finer adjustments for precise positioning.
\end{itemize}

\subsection*{Functions of different components}
\begin{enumerate}
  \item \textbf{Tool Holder}: A tool holder is a device used to secure cutting tools or other attachments in a machine tool. Its primary function is to provide a secure and rigid connection between the tool and the machine, allowing for precise and efficient machining operations. Tool holders often incorporate features such as clamping mechanisms and tool-change systems to facilitate tool installation and removal.

  \item \textbf{Chuck}: A chuck is a specialized type of tool holder used for gripping and holding workpieces in a machine tool, such as a lathe or milling machine. Its main function is to provide a strong and secure grip on the workpiece, allowing for rotational movement or machining operations. Chucks come in various types, such as three-jaw chucks or collet chucks, and can be manually or automatically operated.
  
  \item \textbf{Lead Screw}: A lead screw, also known as a power screw, is a threaded mechanical component used to convert rotary motion into linear motion. It consists of a threaded shaft and a nut that engages with the threads. The primary function of a lead screw is to transmit rotational motion from a motor or handwheel to move a tool or workpiece in a controlled linear manner. Lead screws are commonly used in various machines, such as lathes, milling machines, and CNC systems.
  
  \item \textbf{Feed Rod}: A feed rod, also referred to as a feed handle or feed shaft, is a component that controls the longitudinal movement of a tool or workpiece in a machine tool. Its function is to provide a means for manual or automatic adjustment of the cutting tool's position along the workpiece, allowing for precise feeding or depth-of-cut control. The feed rod is often operated by a handwheel or powered by a motor, and its movement can be coordinated with other machine functions for complex machining operations.
\end{enumerate}   

\subsubsection*{Difference between Lead Screw \& Feed Rod}
Feed rod is used for turning operation and lead screw is used for threading operation in a lathe machine.

\subsection*{Operations}
\begin{itemize}
  \item \textbf{Lathe Machine}: Operations possible on a lathe machine include turning, facing, taper turning, threading, drilling, boring, and parting. Operations not typically possible on a lathe machine include milling, grinding, complex 3D contouring, and gear cutting.
  \item \textbf{Milling Machine}: Milling machine can do all types of operations include face milling, peripheral milling, slot milling, drilling, end milling, chamfer milling, and thread milling. Turning, grinding, boring, taper turning, gear cutting, and parting are not typically possible on a milling machine.
  \item \textbf{Shaper Machine} : Operations possible on a shaper machine include (vertical, horizontal \& inclined) shaping, slotting, planing and only spur gear. Operations not typically possible include drilling, turning, boring, taper turning, other gear cutting, and parting.
  \item \textbf{Drilling Machine} : Drilling machines are used for drilling holes. Operations possible include drilling, counterboring, and countersinking. Operations not typically possible include turning, milling, grinding, boring, taper turning, gear cutting, and parting.
\end{itemize}

\subsubsection*{Types of Motions in machines}
\begin{itemize}
  \item \textbf{Lathe Machine} :  Rotational motion
  \item \textbf{Milling Machine} : Linear and rotary motion
  \item \textbf{Shaping Machine} : Linear reciprocating motion
  \item \textbf{Drilling Machine} : Rotary motion
\end{itemize}
\end{document}
